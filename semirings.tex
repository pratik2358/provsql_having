\section{Testing provenance semantics across different $m-$semirings of interest}
We test our possible-world semantics on different $m-$semirings of interest. We examine whether our semantics make 
sense for absorptive $m-$semirings like the Boolean semiring and the semiring of Boolean functions on Query~\ref{q:agg} which involves very simple aggregate comparison on an imaginary relation $T$ with 
$n$ tuples. We compare the provenance given by our possible-world 
semantics for Query~\ref{q:agg} with the provenance computation for Query~\ref{q:agg}'s \textsc{JOIN}-equivalent version - Query~\ref{q:join}.   

\begin{listing}[h!]
    \label{q:agg}
  \begin{minted}[frame=single,fontsize=\small]{sql}
SELECT DISTINCT 1
  FROM T
 HAVING COUNT(*) = c;
\end{minted}
\end{listing}
\begin{listing}[h!]
    \label{q:join}
\begin{minted}[frame=single,fontsize=\small]{sql}
SELECT DISTINCT 1
  FROM T AS a1
  JOIN T AS a2 ON a1.id <  a2.id
  JOIN T AS a3 ON a2.id <  a3.id
  -- ...
  JOIN T AS a_c ON a_{c-1}.id < a_c.id

EXCEPT

SELECT DISTINCT 1
  FROM T AS b1
  JOIN T AS b2 ON b1.id <  b2.id
  JOIN T AS b3 ON b2.id <  b3.id
  -- ...
  JOIN T AS b_{c+1} ON b_c.id < b_{c+1}.id;
\end{minted}
\end{listing}
We also provide a case study analysis on $m-$semirings where our semantics 
provide different explanations from the provenance semamntics of the join equivalent query.
\subsection{Boolean Semiring}
$$B = (\{0,1\}, \lor, \land, 0, 1)$$
$$\lor : \text{Standard logical OR}$$
$$\land : \text{Standard logical AND}$$
$$0 : \text{Boolean zero}$$
$$1 : \text{Boolean one}$$
\[
\ominus: \text{Monus for Boolean Semiring} \quad a \ominus b := a \land \neg b
\]
Now we take a query as below on a relation $T$ with $n$ tuples:

This returns 1 if the relation $T$ has exactly $c$ tuples, and 0 otherwise.

In possible world semantics, let $w: \{1,2,\dots,m\} \mapsto \{0,1\}$.

The provenance of this query can be expressed as follows:

$$q(T) = \bigoplus_{w}\left(\bigotimes_{w(i)=1}b_i\otimes\left(\mathbb{1}\ominus(\bigoplus_{w(i) =0}b_i)\right)\right)$$

On the other hand, its equivalent join query will be as follows:

The provenance of this query can be expressed as follows:
\[
q'(T) = \bigoplus_{\substack{t\subseteq T \\ |t| = c}}\left(\bigotimes_{\substack{b_i\in t}}b_i\right) \ominus \bigoplus_{\substack{t'\subseteq T \\ |t'| = c+1}}\left(\bigotimes_{\substack{b_i\in t'}}b_i\right)
\]

\begin{proof}
Define
\[
A \;=\; \bigvee_{\lvert t\rvert = c}\;\bigwedge_{i\in t} b_i,
\qquad
B \;=\; \bigvee_{\lvert t'\rvert = c+1}\;\bigwedge_{i\in t'} b_i.
\]
Since in the Boolean semiring \(\oplus=\vee\), \(\otimes=\wedge\) and
\(\;r\ominus s = r\wedge\neg s\), we have
\[
q'(T)
= A \ominus B
= A \wedge \neg B.
\]
Expand step by step:
\begin{align*}
q'(T)
&= \Bigl(\!\bigvee_{\lvert t\rvert = c}\!\bigwedge_{i\in t}b_i\Bigr)
   \;\wedge\;
   \neg\Bigl(\!\bigvee_{\lvert t'\rvert = c+1}\!\bigwedge_{i\in t'}b_i\Bigr)\\[6pt]
&= \Bigl(\!\bigvee_{\lvert t\rvert = c}\!\bigwedge_{i\in t}b_i\Bigr)
   \;\wedge\;
   \Bigl(\!\bigwedge_{\lvert t'\rvert = c+1}\!\neg\bigl(\!\bigwedge_{i\in t'}b_i\bigr)\Bigr)\\[6pt]
&= \bigvee_{\lvert t\rvert = c}
    \Bigl(
      \bigwedge_{i\in t}b_i
      \;\wedge\;
      \bigwedge_{|t'|=c+1}\!\neg\bigl(\!\bigwedge_{i\in t'}b_i\bigr)
    \Bigr).
\end{align*}

\textbf{Key simplification.}  Fix a particular \(t\) with \(\lvert t\rvert=c\).  Consider
\[
\bigwedge_{\substack{t'\subseteq[m]\\|t'|=c+1}}
\neg\Bigl(\!\bigwedge_{i\in t'}b_i\Bigr).
\]
- For each \(j\notin t\), the subset \(t'=t\cup\{j\}\) yields
  \(\neg\bigl(b_j\wedge\bigwedge_{i\in t}b_i\bigr)=\neg b_j\).  
- Any other \(t'\) also contains some \(j\notin t\), so its factor
  \(\neg(\bigwedge_{i\in t'}b_i)\) can be written as
  \(\neg b_j\vee Y\).  

with \(x=\neg b_j\), each such pair of factors collapses to \(\neg b_j\).  Hence the entire
AND reduces to
\(\bigwedge_{j\notin t}\neg b_j\).

Substituting back gives
\[
q'(T)
=\bigvee_{\lvert t\rvert = c}
\Bigl(\,\bigwedge_{i\in t}b_i
       \;\wedge\;
       \bigwedge_{j\notin t}\neg b_j
\Bigr)
=\;q(T),
\]
as required.
\end{proof}
\subsection{Boolean‐Function Semiring}

Let \(X\) be a set of worlds.  
Define the \emph{Boolean‐function m‐semiring}:
\[
  \mathcal{B}(X)
  = \bigl(\{f : X \to \{0,1\}\},\oplus,\otimes,0_{\mathcal{B}},1_{\mathcal{B}},\delta,\ominus\bigr),
\]
where, for \(f,g : X\to\{0,1\},\;x\in X\):
\[
  (f\oplus g)(x)=f(x)\vee g(x), \quad (f\otimes g)(x)=f(x)\wedge g(x),
\]
\[
  (f\ominus g)(x)=f(x)\wedge\neg g(x),\quad \delta(f)(x)=
  \begin{cases}
  1 & \text{if } f(x)=1,\\
  0 & \text{otherwise.}
  \end{cases}
\]
Here \(0_{\mathcal{B}}\) and \(1_{\mathcal{B}}\) are the constant‐0 and constant‐1 functions.  
Let the additive monoid of aggregate values be \((M,+_M,\mathbb{0}_M)=(\mathbb{N},+,0)\).

---

\paragraph{Aggregation Semantics.}  
An aggregation expression is
\[
\mathbf{a}=\bigoplus_{\mathcal{M},i}(a_i^{(K)},a_i^{(M)}),
\]
with \(a_i^{(K)}\in\mathcal{B}(X)\) the Boolean provenance annotation and \(a_i^{(M)}\in\mathbb{N}\) its numeric monoid value.  

Given two aggregations \(\mathbf{u},\mathbf{v}\), their comparison is:
\begin{multline}
[\mathbf{u}\oslash \mathbf{v}]_{\mathrm{op}}=
\bigoplus_{S,T}
\Biggl(
\Bigl(\bigotimes_{i\in S}u_i^{(K)}\Bigr)
\otimes
\Bigl(\bigotimes_{j\in T}v_j^{(K)}\Bigr) \\
\otimes\;\chi_{\mathrm{op}}\bigl(\mathrm{val}_\mathbf{u}(S),\mathrm{val}_\mathbf{v}(T)\bigr)
\otimes
\Biggl(1_{\mathcal{B}}\ominus\Bigl(\bigoplus_{i\notin S}u_i^{(K)}\;\oplus\;\bigoplus_{j\notin T}v_j^{(K)}\Bigr)\Biggr)
\Biggr),
\end{multline}
where
\[
\chi_{\mathrm{op}}(x,y)=
\begin{cases}
1_{\mathcal{B}} & \text{if } x\;\mathit{op}\; y, \\
0_{\mathcal{B}} & \text{otherwise,}
\end{cases}
\]
and
\[
\mathrm{val}_\mathbf{u}(S)=\sum_{i\in S}u_i^{(M)},\quad
\mathrm{val}_\mathbf{v}(T)=\sum_{j\in T}v_j^{(M)}.
\]

---

\paragraph{SQL Example.}  
On a relation \(T\) with Boolean annotations \(b_1,\dots,b_n\in\mathcal{B}(X)\):
Its provenance is
\[
q(T)=\bigoplus_{w:\{1,\dots,n\}\to\{0,1\}}
\Bigl(
\bigotimes_{w(i)=1}b_i
\;\otimes\;
(1_{\mathcal{B}}\ominus\bigoplus_{w(i)=0}b_i)
\Bigr).
\]

Its provenance is
\[
q'(T)=\Bigl(\!\bigoplus_{t\subseteq[n],|t|=c}\bigotimes_{i\in t}b_i\Bigr)
\ominus
\Bigl(\!\bigoplus_{t'\subseteq[n],|t'|=c+1}\bigotimes_{i\in t'}b_i\Bigr).
\]

---

\begin{proposition}
For all \(T\), \(q'(T)=q(T)\) in \(\mathcal{B}(X)\).
\end{proposition}

\begin{proof}
Define
\[
A=\bigoplus_{|t|=c}\bigotimes_{i\in t}b_i, \quad
B=\bigoplus_{|t'|=c+1}\bigotimes_{i\in t'}b_i.
\]
Then
\[
q'(T)=A\ominus B=A\otimes(\neg B).
\]
All operations are pointwise in \(\mathcal{B}(X)\), so for each world \(x\in X\):
\[
q'(T)(x)=A(x)\wedge\neg B(x).
\]
This matches the per-world condition in the $\oslash$-semantics of $q(T)$, so $q'(T)=q(T)$.
\end{proof}

\noindent\textbf{Explanation.}
\begin{itemize}
  \item Domain: Boolean functions over $X$, not just $\{0,1\}$.  
  \item Operations: $\oplus,\otimes,\ominus$ act pointwise as $\vee,\wedge,\wedge\neg$.  
  \item Because $\mathcal{B}(X)$ is idempotent and absorptive, join-except coincides with world-sum provenance.
\end{itemize}              
\subsection{Homomorphisms}
We have seen that our provenance semantics work for 
the semiring of Boolean functions. In this section we investigate other m-semirings and see if they are homomorphic 
to the semiring of Boolean functions. This will ensure that the provenance sematics, that worked for $B[X]$ can be transferred to 
our target semirings without re-deriving from scratch.


\subsubsection{$\Phi^{\mathrm{T}}_v : B[X] \to T$}
For any valuation $v:X \to T$,  $\exists$ a homomorphism $\Phi^{\mathrm{T}}_v : B[X] \to T$ such that for all $x\in X$, 
$\Phi^{\mathrm{T}}_v(x\mapsto [x]_{B[X]})=v(x)$.

\subsubsection*{Existence(by construction)}
\begin{proof}
Let $\mathsf{Form}(X)$ be the set of well formed Boolean formulas built on the variables in X. There is a canonical surjection 
$q:\mathsf{Form}(X)\rightarrow B[X]$(basically maps each syntactical formula to its equivalence class).

We define $\iota: X \rightarrow \mathsf{Form}(X)$, and thus essentially any Boolean function in $B[X]$ is a composition $q\circ \iota$.

Define a map $\Phi^*:\mathsf{Form}(X)\rightarrow T$ by recursion:
\begin{align*}
&\Phi^*(0_B) := \varnothing,\quad\Phi^*(1_B) := \mathbb{R},\\
&\Phi^*(\iota(x)) := v(x)\quad\text{for }x\in X,\\
\vspace{2cm}
&\text{And for any }\varphi,\psi \in \mathsf{Form}(X):\\
&\Phi^*(\varphi\vee\psi) := \Phi^*(\varphi)\cup\Phi^*(\psi),\\
&\Phi^*(\varphi\wedge\psi) := \Phi^*(\varphi)\cap\Phi^*(\psi),\\
&\Phi^*(\neg\varphi) := \mathbb{R}\setminus\Phi^*(\varphi)= \Phi^*(\varphi)^\complement\\
&\Phi^*(\phi \setminus_B \psi) = \Phi^*(\phi \wedge \neg \psi) = \Phi^*(\phi)\cap \Phi^*(\neg \psi)
\end{align*}
such that the Boolean connectives are interpreted in $T$ through the set-theoretic operations $\cup, \cap$ and set complement.

$\mathsf{Form}(X)$ being inductively generated, $\Phi^*$ is well-defined on all formulae.
  For any two logically equivalent formulae $\phi, \psi \in \mathsf{Form}(X)$ we have, $\Phi^*(\phi)=\Phi^*(\psi)$ in $T$. This is because every Boolean algebra axiom in $Form(X)$  is interpreted in $T$ through set-theoretic operations and since
  $\mathsf{ker}(\Phi^*):=\{(\phi,\psi)\in \mathsf{Form}(X)\times \mathsf{Form}(X)|\Phi^*(\phi)=\Phi^*(\psi)\}$ is a Boolean congrunence satisfying the equational theory of Boolean algebra, 
  therefore logically equivalent formulae are mapped to the same element in $T$ by $\Phi^*$.
  Thus for any Boolean formula $\phi$, we can define a homomorphism on it's equivalence class(which is essentially a Boolean function) as,  $\Phi_v^T(q(\phi)):=\Phi^*(\phi)$. This ensures that $\Phi_v^T$ is well-defined and for 
  any variable $x \in X$, $\Phi_v^T(x\mapsto [x]_{B[X]})=\Phi_v^T(q(\iota(x)))=\Phi^*(\iota(x))=v(x)$.

\end{proof}

\subsubsection*{Uniqueness}
\begin{proof}
Assume that $\exists\;\Psi_v^T:B[x]\rightarrow T$ such that $\Psi_v^T(x\mapsto [x]_{B[X]})=v(x)\text{ for all }x \in X$.

$B[X]$ is a free Boolean algebra on $X$  and $\alpha \in B[X]$ is image of $q:\mathsf{Form}(X)\rightarrow B[X]$ of some boolean formula $\phi$. 
When it comes to free algebras, homomorphisms are only determined by the values of their generators(all Boolean atomic functions, $[x]_{B[x]}\in B[X]$)
And by assumption of $\Psi_v^T$ and existence of $\Phi_v^T$, they agreee on all generators. Hence, $\Psi_v^T=\Phi_v^T$.
\end{proof}

\subsubsection*{$\Phi_v^T$ preserves Axiom A.13.}

Let $\phi, \psi, \gamma \in B[X]$ and pick representatives $\tilde{\phi}, \tilde{\psi}, \tilde{\gamma} \in \mathsf{Form}(X)$ such that
\[
q(\tilde{\phi}) = \phi, \quad q(\tilde{\psi}) = \psi, \quad q(\tilde{\gamma}) = \gamma.
\]

Then we have
\[
\begin{aligned}
\Phi_v^T\big(\phi \wedge (\psi \setminus_B \gamma)\big) 
&= \Phi^*\big(\tilde{\phi} \wedge (\tilde{\psi} \wedge \neg \tilde{\gamma})\big) \\
&= \Phi^*(\tilde{\phi}) \cap \Phi^*(\tilde{\psi} \wedge \neg \tilde{\gamma}) \\
&= \Phi^*(\tilde{\phi}) \cap \big(\Phi^*(\tilde{\psi}) \cap \Phi^*(\neg \tilde{\gamma})\big) \\
&= \Phi^*(\tilde{\phi}) \cap \Phi^*(\tilde{\psi}) \cap (\Phi^*(\tilde{\gamma}))^\complement.
\end{aligned}
\]

Similarly,
\[
\begin{aligned}
\Phi_v^T\big((\phi \wedge \psi) \setminus_B (\phi \wedge \gamma)\big) 
&= \Phi^*\big((\tilde{\phi} \wedge \tilde{\psi}) \wedge \neg (\tilde{\phi} \wedge \tilde{\gamma})\big) \\
&= (\Phi^*(\tilde{\phi}) \cap \Phi^*(\tilde{\psi})) \cap (\Phi^*(\tilde{\phi} \wedge \tilde{\gamma}))^\complement \\
&= (\Phi^*(\tilde{\phi}) \cap \Phi^*(\tilde{\psi})) \cap (\Phi^*(\tilde{\phi}) \cap \Phi^*(\tilde{\gamma}))^\complement \\
&= (\Phi^*(\tilde{\phi}) \cap \Phi^*(\tilde{\psi})) \cap (\Phi^*(\tilde{\phi})^\complement \cup \Phi^*(\tilde{\gamma})^\complement) \\
&= (\underbrace{\Phi^*(\tilde{\phi}) \cap \Phi^*(\tilde{\psi}) \cap \Phi^*(\tilde{\phi})^\complement}_{\varnothing}) 
   \;\cup\; (\Phi^*(\tilde{\phi}) \cap \Phi^*(\tilde{\psi}) \cap \Phi^*(\tilde{\gamma})^\complement) \\
&= \Phi^*(\tilde{\phi}) \cap \Phi^*(\tilde{\psi}) \cap (\Phi^*(\tilde{\gamma}))^\complement.
\end{aligned}
\]

Hence,
\[
\Phi_v^T\big(\phi \wedge (\psi \setminus_B \gamma)\big) = \Phi_v^T\big((\phi \wedge \psi) \setminus_B (\phi \wedge \gamma)\big),
\]
which shows that $\Phi_v^T$ preserves Axiom A13.


\subsubsection{$\Phi_{\mathrm{trop}}: B^X \to T_{\min}$}

We define:
$$
T_{\min} = (\mathbb{R}_{\ge 0} \cup \{\infty\},\min,+,\infty,0,\setminus_T).
$$
\begin{itemize}
    \item Addition: \(\min\)
    \item Multiplication: \(+\)
    \item Zero: \(\infty\)
    \item One: \(0\)
    \item Monus:
    \[
    a \setminus_T b :=
    \begin{cases}
        a & \text{if } a < b \\
        \infty & \text{otherwise}
    \end{cases}
    \]
\end{itemize}



\textbf{Homomorphism}:\newline
Assign each token \(x_i\in X\) a cost \(c(x_i)\in\mathbb{R}_{\ge0}\cup\{\infty\}\). For any valuation \(\nu\), define
$$
w(\nu) = \sum_{x_i:\nu(x_i)=\top} c(x_i).
$$
Then for boolean function  \(\varphi\):
$$
\Phi_{\mathrm{trop}}(\varphi) 
= \min_{\nu:\varphi(\nu)=\top} w(\nu),
$$
with \(\Phi_{\mathrm{trop}}(0_B)=\infty\).
\newline

\textbf{Proof sketch}: \newline
- \(\Phi_{\mathrm{trop}}(\varphi \vee \psi) = \min(\Phi_{\mathrm{trop}}(\varphi), \Phi_{\mathrm{trop}}(\psi))\).  
- \(\Phi_{\mathrm{trop}}(\varphi \wedge \psi) = \Phi_{\mathrm{trop}}(\varphi) + \Phi_{\mathrm{trop}}(\psi)\).  
- Monus: follows. \newline
\textbf{Example}: \newline
Let \(X=\{x_1,x_2\}\), \(c(x_1)=2\), \(c(x_2)=5\).  
- \(\varphi = x_1\): \(\Phi_{\mathrm{trop}}(\varphi)=2\).  
- \(\psi = x_2\): \(\Phi_{\mathrm{trop}}(\psi)=5\).  
- \(\varphi \setminus_B \psi\) true only when \(\nu=(\top,\bot)\), cost \(2\).

It is easy to show that $A13$ and absorptivity are satisfied.


\subsubsection{$\Phi_{\mathrm{sec}}: B^X \to T_{\sec}$}

For each \(x\in X\) , say, there is a clearance level \(\lambda(x)\in \mathcal L\) assigned to it,
with \(\mathcal L\) totally ordered: \(\ell_{\min}\) (least restrictive),
and \(\ell_{\max}\) (most restrictive).

For a valuation \(\nu\):
\[
S(\nu) = \{x\in X \mid \nu(x)=\top\}.
\]
\[
\sec(\nu) =
\begin{cases}
\ell_{\min}, & S(\nu)=\emptyset,\\
\max_{x\in S(\nu)} \lambda(x), & \text{o.w.}
\end{cases}
\]



Then we can define a homomorphism:

\[
\Phi_{\mathrm{sec}}(\varphi)
\;=\;
\min_{\nu:\varphi(\nu)=\top}\;\sec(\nu),
\]
and if \(\varphi\equiv 0\), then
\[
\Phi_{\mathrm{sec}}(0_B) = \ell_{\max}.
\]

This is basically the “minimum clearance level required to get a tuple as a result.”



\textbf{Proof sketch:}
\begin{itemize}
    \item \textbf{Zero and One}  
        \begin{itemize}
            \item \(\Phi_{\mathrm{sec}}(0_B)=\ell_{\max}\) (no world satisfies 0).
            \item \(\Phi_{\mathrm{sec}}(1_B)=\ell_{\min}\) (world with no tokens gives \(\ell_{\min}\)).
        \end{itemize}
    \item \textbf{Addition} (\(\vee \to \min\))  
        \[
            \Phi(\varphi\vee\psi)
            = \min\bigl(\Phi(\varphi),\Phi(\psi)\bigr).
        \]
    \item \textbf{Multiplication} (\(\wedge \to \max\))  
        
        \[
            \Phi(\varphi\wedge\psi)
            = \max\bigl(\Phi(\varphi),\Phi(\psi)\bigr).
        \]
    \item \textbf{Monus} (\(\setminus\))  
        Set of valuations for which  \(\varphi\setminus_B\psi\) holds =  
        \(\{\nu \mid \varphi(\nu)=\top,\;\psi(\nu)=\bot\}\).  
        We can take \(\min\) of their security levels.
        Monus of security semiring?????
        \[
            a\setminus_S b =
            \begin{cases}
                a, & a<b,\\
                \ell_{\max}, &\text{otherwise}.
            \end{cases}
        \]
\end{itemize}

---

 5. Smalll Example

Let \(X=\{x,y\}\) with \(\lambda(x)=\mathrm{Pub}\), \(\lambda(y)=\mathrm{Sec}\),
and ordering \(\mathrm{Pub}<\mathrm{Sec}<\mathrm{Top}\).

- \(\varphi = x\):  
  Worlds: \((1,0)\to\mathrm{Pub}\), \((1,1)\to\mathrm{Sec}\).  
  \(\Phi(x)=\min(\mathrm{Pub},\mathrm{Sec})=\mathrm{Pub}.\)

- \(\psi = y\):  
  Worlds: \((0,1),(1,1)\to\mathrm{Sec}\).  
  \(\Phi(y)=\mathrm{Sec}.\)

- \(\varphi\vee\psi\): worlds \((1,0),(0,1),(1,1)\) → clearances Pub, Sec, Sec → min = Pub.

- \(\varphi\wedge\psi\): only \((1,1)\) → clearance max(Pub, Sec)=Sec.

- \(\varphi\setminus_B\psi\): world \((1,0)\) only → Pub, matching \(\mathrm{Pub}\setminus_S\mathrm{Sec}=\mathrm{Pub}.\)



\subsection{counting semiring}
\begin{toappendix}

\section*{Counting Semiring Definition}

\[
k = (\mathbb{N}, \oplus, \otimes, 0, 1)
\]

where:

\[
\mathbb{N} = \{0, 1, 2, \dots\}
\]

\[
\begin{aligned}
\oplus &: \text{standard integer addition}, \\
\otimes &: \text{standard integer multiplication}, \\
0 &\in \mathbb{N} : \text{zero element}, \\
1 &\in \mathbb{N} : \text{one element}.
\end{aligned}
\]

---

\subsection*{Additional Operators}

\paragraph{Monus (Truncated Subtraction):}
\[
a \ominus b = 
\begin{cases}
0, & a \le b, \\
a - b, & a > b.
\end{cases}
\]

\paragraph{Delta Function:}
\[
\delta(a) = 
\begin{cases}
0, & a = 0, \\
1, & a > 0.
\end{cases}
\qquad
\delta : k \mapsto k
\]
\[
\delta(0_k) = 0_k, \quad \delta(n \cdot 1_k) = 1_k
\]

---

\subsection*{Counting Semiring Properties}

\paragraph{1. Additive Monoid}
\[
\begin{aligned}
(a \oplus b) \oplus c &= a \oplus (b \oplus c) && \text{(Associativity)} \\
a \oplus b &= b \oplus a && \text{(Commutativity)} \\
a \oplus 0 &= a && \text{(Identity)}
\end{aligned}
\]

\paragraph{2. Multiplicative Monoid}
\[
\begin{aligned}
(a \otimes b) \otimes c &= a \otimes (b \otimes c) && \text{(Associativity)} \\
a \otimes b &= b \otimes a && \text{(Commutativity)} \\
a \otimes 1 &= a && \text{(Identity)}
\end{aligned}
\]

\paragraph{3. Distributivity}
\[
\begin{aligned}
a \otimes (b \oplus c) &= (a \otimes b) \oplus (a \otimes c) && \text{(Left Distributivity)} \\
(a \oplus b) \otimes c &= (a \otimes c) \oplus (b \otimes c) && \text{(Right Distributivity)}
\end{aligned}
\]

\paragraph{4. Absorption}
\[
a \otimes 0 = 0 \otimes a = 0
\]

\paragraph{5. Monus Properties}
For all \( a, b \in k \):
\[
\begin{aligned}
(a \ominus b) \oplus b &= a, && \text{if } a \ge b, \\
a \ominus 0 &= a, \\
0 \ominus a &= 0.
\end{aligned}
\]

\paragraph{6. Semimodule Operation}
\[
\begin{aligned}
a \otimes (b * m) &= (a \otimes b) * m && \text{(Associativity with multiplication)} \\
1 * m &= m && \text{(Identity)} \\
0 * m &= 0 && \text{(Absorption)}
\end{aligned}
\]

\paragraph{7. Other Properties}
\[
\delta(u) \otimes [[u \oslash v]]_{\textit{op}} =
\begin{cases}
1, & \text{if } (u \ \textit{op}\ v) \text{ holds in } \mathbb{N}, \\
0, & \text{otherwise.}
\end{cases}
\]

---

\subsection*{Example: COUNT Aggregation with HAVING Clause}

For a COUNT(*) aggregate, let:
\[
f = \bigoplus_{g \in \text{group}} 1
\]

Then to filter using a HAVING condition (e.g., \( f > C \)):

\[
\delta(f) \otimes [[f \oslash C]]_{>} =
\begin{cases}
1, & \text{if } f > C \text{ in } \mathbb{N}, \\
0, & \text{otherwise.}
\end{cases}
\]

\end{toappendix}


Let
\[
  \mathcal{N} = (\mathbb{N},+,\cdot,0,1)
\]
be the \emph{counting semiring}, and equip it with the \emph{monus} (truncated subtraction)
\[
  a \ominus b =\max(a - b,0).
\]

On a relation \(T\) with annotations \(b_1,\dots,b_n\in\mathbb{N}\), consider the two queries of 
“\(\mathrm{COUNT}(*) = c\)” exactly as before:



Its provenance in \(\mathcal{N}\) becomes
\[
  q(T)
  =
  \bigoplus_{W}
    \Bigl(\bigotimes_{i\in W\cap T}b_i\Bigr)\otimes\chi_{=}\Bigl({\bigplus_{\mathbb{N}}}_{i\in W\cap T} 1, c\Bigr)\otimes\Bigl(1 \ominus \bigl(\bigoplus_{j\in W\cap T}b_j\bigr)\Bigr)
  =
  \sum_{W\subseteq T}
    \Bigl(\prod_{i\in W\cap T}b_i\cdot\bigl(1\ominus\sum_{j\in W\cap T}b_i\bigr)\Bigr).
\]

Its provenance is
\[
  q'(T)
  =
  \Bigl(\sum_{\substack{t\subseteq[n],\\|t|=c}}\prod_{i\in t}b_i\Bigr)
  \ominus
  \Bigl(\sum_{\substack{t'\subseteq[n],\\|t'|=c+1}}\prod_{i\in t'}b_i\Bigr).
\]

\begin{proposition}
There exist annotations and a choice of \(c\) for which
\(q'(T)\neq q(T)\)
in the counting semiring \(\mathcal{N}\).
\end{proposition}

\begin{proof}
Take \(\lvert T\rvert=2\), \(c=1\), and annotations
\[
  b_1 = 1,\quad b_2 = 2.
\]
Then the only world contributing to \(q(T)\) is \(w(1)=w(2)=1\), since any world with an absent tuple gives
\((1\ominus\sum_{w(i)=0}b_i)=0\).  Hence
\[
  q(T)
  = (1\cdot2)\bigl(1\ominus0\bigr)
  = 2.
\]
On the other hand,
\[
  q'(T)
  = (b_1+b_2)\ominus(b_1\cdot b_2)
  = (1+2)\ominus(1\cdot2)
  = 3\ominus2
  = 1.
\]
Thus \(q'(T)=1\neq2=q(T)\).
\end{proof}

\bigskip
\noindent\textbf{Discussion of the Failure}

\begin{itemize}
  \item In \(\mathcal{N}\), addition and multiplication are \emph{not} idempotent, nor is monus an involution.
  \item The Boolean proof relied on \emph{absorptivity} and pointwise idempotence (\(1\vee1=1\)), which fail for \(\mathbb{N}\).
  \item Consequently the “join‐except” provenance
    \(\sum_{|t|=c}-\sum_{|t'|=c+1}\)
    does \emph{not} coincide with the summation over worlds
    \(\sum_{w}\prod_{w(i)=1}n_i\cdot(1\ominus\sum_{w(i)=0}n_i)\).
\end{itemize}



