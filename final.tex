\documentclass{scrartcl}
\usepackage{amssymb,amsfonts,amsmath,amsthm}
\usepackage{paralist}

\title{On the Semantics of the Provenance of HAVING queries in Semirings
with Monus}

\renewcommand{\epsilon}{\varepsilon}
\renewcommand{\leq}{\leqslant}
\renewcommand{\geq}{\geqslant}

\newtheorem{corollary}{Corollary}
\newtheorem{proposition}{Proposition}
\newtheorem{theorem}{Theorem}
\newcommand\op{\star}
\newcommand\complexity[1]{\textsf{\upshape #1}}

\begin{document}
\maketitle

\section{Introduction}

\section{Preliminaries}

\section{Semantics}

\section{Results}

\begin{proposition}
  Let $S$ be any m-semiring. $S$ is idempotent if and only if $\ominus$
  is right-distributive over~$\oplus$ in $S$.
\end{proposition}
\begin{proof}
  \textbf{($\Rightarrow$)}
  Assume S to be additively idempotent. Define a preorder $\leq$ such that,
  if $a\leq b$ \text{then }$\exists x,\; a\oplus x=b$.
  So, $a \oplus b = a \oplus (a \oplus x)=(a\oplus a)\oplus x=a \oplus x = b$. \\$\therefore a\leq b \implies a\oplus b=b$.
  \\Conversely, if $a\oplus b = b$, choose $x=b$. And trivially, $a\oplus x =a\oplus b=b$
  $\implies a \leq b$.
  \\$\therefore a\leq b \iff a \oplus b =b$. Therefore, $(S,\leq)$ is indeed a partial order and $\oplus$ is monotone under $\leq$.
  \\Now, $\forall u, a \leq u$ and $b \leq u$
  \\ $a \oplus u = u$ and $b\oplus u =u$.
  \\But, $a\leq a \oplus b$ and $b \leq a \oplus b$
  \\$\therefore a \oplus b $ is the least such $u$ for which $a\leq u $ and $b \leq u$.
  So, $\oplus$ is the join of a semilattice of $(S,\leq)$.

  \paragraph{i.}
  By definition of monus : $x \ominus y \leq z \iff x \leq y \oplus z$, 
  $a\leq a\oplus b\leq c \oplus ((a\oplus b)\ominus c) \implies a\ominus c\leq (a\oplus b)\ominus c$.
  Similarly, $b\ominus c\leq (a\oplus b)\ominus c$.
  \\$\therefore (a\ominus c)\oplus (b\ominus c) \leq (a\oplus b) \ominus c$
  \paragraph{ii.} We have $a\leq c \oplus (a\ominus c)$ and $b\leq c\oplus (b\ominus c)$. So, $a\oplus b \leq c\oplus c \oplus (a\ominus c) \oplus (b\ominus c)$.
  By idempotence of $\oplus$,  $a\oplus b \leq c \oplus (a\ominus c) \oplus (b\ominus c)$.
\\$\therefore $ $(a\oplus b) \ominus c \leq (a\ominus c)\oplus (b\ominus c) $, by definition of monus.
 \\ From \textbf{i.} \& \textbf{ii.}
  $$\boxed{(a\oplus b) \ominus c = (a\ominus c)\oplus (b\ominus c)}$$
\textbf{($\Leftarrow$)}
Assume that $\ominus$ of S is right-distributive:
\[
  (a\oplus b)\ominus c
  = (a\ominus c)\oplus(b\ominus c)\;\forall a,b,c\in S.
\]

For any arbitrary $a\in S$ substitute $b=a$ and $c=a$:
\[
  (a\oplus a)\ominus a
  = (a\ominus a)\oplus(a\ominus a).
\]
Since $a\ominus a=0$, the right-hand side is $0$. And $(a\oplus a)\ominus a = 0\implies a\oplus a\leq a$ by definition of monus.
  But, trivially $a\le a\oplus a$, so antisymmetry of $(S,\leq)$ gives $a\oplus a=a$.
  Thus, $$\boxed{a \oplus a=a, \; \forall a\in S}$$
\smallskip
Both sides, prove the equivalence.
\end{proof}
% This is wrong but why?
%\begin{proposition}
%   Let $K_1, K_2$ be two m-semirings and $h: K_1\to K_2$ a semiring
%   homomorphism. Then $h$ is an m-semiring homomorphism.
% \end{proposition}

\begin{proposition}
  Let $K_1, K_2$ be two m-semirings and $h: K_1\to K_2$ an m-semiring
  homomorphism. Then:
    \begin{compactenum}[(i)]
      \item If $\mathbb{K}_1$ is idempotent, then $\mathbb{K}_2$ is
        idempotent.
      \item If $\otimes$ is distributive over $\ominus$ in $K_1$, the same
        is true in~$\mathbb{K}_2$.
    \end{compactenum}
\end{proposition}

\begin{corollary}
  The following m-semirings are idempotent and have distributivity of
  $\otimes$ over $\ominus$: the temporal semiring, the Bool[X] semiring,
  the Which[X] semiring, etc.
\end{corollary}

\begin{theorem}
  For any binary query $q$, for any $C\in\mathbb{N}$, for $\op\in \{\leq,
  =,\geq\}$ consider the queries:
  \begin{align*}
    Q^{\op C}_1&=\Pi_{\#1}(\sigma_{\#2\op C}(\gamma_{\#1}[1:+](q(R))))\\
  Q^{\geq C}_2&=\Pi_{\#1}(q(R)\bowtie_{\#1=\#3\land\#2<\#4}
    q(R)\bowtie_{\#1=\#5\land\#4<\#6}\dots\bowtie_{\#1=\#(2C+1)\land\#(2C)<\#(2C+2)}q(R))\\
Q^{\leq C}_2&=q(R)-Q_2^{\geq C+1}\\
Q^{=C}_2&=Q_2^{\geq C} -Q_2^{\geq C+1}
\end{align*}
    Then:
    \begin{compactenum}[(i)]
      \item
        For any commutative m-semiring $\mathbb{K}$ which is
        idempotent and such that $\otimes$ is
        distributive over $\ominus$, for any
        $\mathbb{K}$-instance $\hat I$,
        $\langle Q^{\op C}_1\rangle^{\hat I}=\langle Q^{\op C}_2\rangle^{\hat I}$.
      \item There exists a commutative m-semiring $\mathbb{K}_1$ which is
        idempotent but does
        not have distributivity of $\otimes$ over $\ominus$, and a
        $\mathbb{K}_1$-instance $\hat I_1$ such that
        $\langle Q^{\op C}_1\rangle^{\hat I_1}\neq\langle Q^{\op C}_2\rangle^{\hat I_1}$.
      \item There exists a commutative m-semiring $\mathbb{K}_2$ that has
        distributivity of $\otimes$ over $\ominus$ but which is not
        idempotent, and a
        $\mathbb{K}_2$-instance $\hat I_2$ such that
        $\langle Q^{\op C}_1\rangle^{\hat I_2}\neq\langle Q^{\op C}_2\rangle^{\hat I_2}$.
    \end{compactenum}
\end{theorem}
\begin{proof}

Let $U:=\mathrm{supp}(\textsc{COUNT(*)})$. For any $W\subseteq U$ define the monomial
\[
A_W := \bigotimes_{x\in W}\mathbf{x}^{(K)},
\]
and for $|W|=c$ define the ``exact-$W$'' contribution
\[
t_W := A_W \otimes \Bigl(\mathbb{1}_K \ominus \bigoplus_{x\in U\setminus W}\mathbf{x}^{(K)}\Bigr).
\]
Put
\[
S_c := \bigoplus_{\substack{W\subseteq U\\|W|=c}} A_W,\qquad
S_{c+1} := \bigoplus_{\substack{W'\subseteq U\\|W'|=c+1}} A_{W'},
\qquad
\bigl[\textsc{COUNT(*)}\oslash c\bigr]_{=}
   := \bigoplus_{\substack{W\subseteq U\\|W|=c}} t_W.
\]


We first prove by induction on $n\ge 1$ the identity
\[
\Bigl(\bigoplus_{i=1}^n a_i\Bigr)\ominus c
   = \bigoplus_{i=1}^n (a_i\ominus c),
\tag{I}
\]
for arbitrary $a_1,\dots,a_n,c\in K$.

Base $n=1$ is trivial. For $n=2$ this is exactly A14:
\[
(a_1\oplus a_2)\ominus c = (a_1\ominus c)\oplus(a_2\ominus c).
\]
Assume (I) holds for $n-1$. Then
\[
\begin{aligned}
\Bigl(\bigoplus_{i=1}^n a_i\Bigr)\ominus c
&= \Bigl(\bigl(\bigoplus_{i=1}^{\,n-1} a_i\bigr)\oplus a_n\Bigr)\ominus c \\
&\overset{\text{A14}}{=}
\Bigl(\bigoplus_{i=1}^{\,n-1} a_i\Bigr)\ominus c \;\oplus\; (a_n\ominus c) \\
&\overset{\text{IH}}{=}
\bigoplus_{i=1}^{\,n-1} (a_i\ominus c) \;\oplus\; (a_n\ominus c)
= \bigoplus_{i=1}^n (a_i\ominus c).
\end{aligned}
\]
Thus (I) holds for every finite $n$.

Using the induction result (I) with the outer sum in $S_c$ and $c:=S_{c+1}$ we obtain
\[
S_c\ominus S_{c+1}
= \Bigl(\bigoplus_{|W|=c} A_W\Bigr)\ominus S_{c+1}
= \bigoplus_{|W|=c} \bigl(A_W\ominus S_{c+1}\bigr).
\tag{1}
\]
Every $(c{+}1)$-subset $W'$ contains exactly one $c$-subset.  
Thus
\[
S_{c+1}
= \bigoplus_{\substack{W\subseteq U\\|W|=c}}\;
   \bigoplus_{u\in U\setminus W} A_{W\cup\{u\}}.
\]
  Using (I):
\[
S_c \ominus S_{c+1}
=
\bigoplus_{\substack{W\subseteq U\\|W|=c}}
\Bigl(
A_W \;\ominus\;
\bigoplus_{u\in U\setminus W} A_{W\cup\{u\}}
\Bigr).
\tag{$\ast$}
\]


Fix $W\subseteq U$ with $|W|=c$ and put
\[
R_W := \bigoplus_{x\in U\setminus W}\mathbf{x}^{(K)}.
\]
Since $A_{W\cup\{u\}}=A_W\otimes\mathbf{u}^{(K)}$ for every $u\in U\setminus W$, distributivity of $\otimes$ over $\oplus$ yields
\[
\bigoplus_{u\in U\setminus W} A_{W\cup\{u\}}
= \bigoplus_{u\in U\setminus W} \bigl(A_W\otimes\mathbf{u}^{(K)}\bigr)
= A_W \otimes \Bigl(\bigoplus_{u\in U\setminus W}\mathbf{u}^{(K)}\Bigr)
= A_W\otimes R_W.
\]
Now apply right-distributivity of $\otimes$ over $\ominus$ in the form
\[
(a\ominus b)\otimes c = (a\otimes c)\ominus(b\otimes c),
\]
with $a=\mathbb{1}_K,\; b=R_W,\; c=A_W$. This gives
\[
(\mathbb{1}_K\ominus R_W)\otimes A_W
= A_W \ominus (A_W\otimes R_W).
\]
Using commutativity of $\otimes$ we obtain
\[
A_W \ominus \bigoplus_{u\in U\setminus W} A_{W\cup\{u\}}
= A_W \ominus (A_W\otimes R_W)
= A_W \otimes(\mathbb{1}_K\ominus R_W)
= t_W.
\]
Substituting this equality into $(\ast)$ and summing over all $W$ with $|W|=c$ yields
\[
S_c\ominus S_{c+1}
= \bigoplus_{\substack{W\subseteq U\\|W|=c}} t_W,
\]
as required.

\end{proof}
\begin{proposition}
  For any binary query $q$, for any $C\in\mathbb{N}^*$, for $\op\in \{\leq,
  =,\geq\}$, for any commutative $m$-semiring $\mathbb{K}$,
  computing
  $\langle Q^{\op C}\rangle^{\hat I}$ for the query
\[    Q^{\op C}_1=\Pi_{\#1}(\sigma_{\#2\op C}(\gamma_{\#1}[1:+](q(R))))
\] over a $\mathbb{K}$-instance $\hat I$ is \complexity{\#P}-hard.
\end{proposition}

\section{Implementing in Practice}

\section{Experiments}

\section{Conclusion}

\end{document}
