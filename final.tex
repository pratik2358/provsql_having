\documentclass{scrartcl}
\usepackage{amssymb,amsfonts,amsmath}
\usepackage{paralist}

\title{On the Semantics of the Provenance of HAVING queries in Semirings
with Monus}

\renewcommand{\epsilon}{\varepsilon}
\renewcommand{\leq}{\leqslant}
\renewcommand{\geq}{\geqslant}

\newtheorem{corollary}{Corollary}
\newtheorem{proposition}{Proposition}
\newtheorem{theorem}{Theorem}
\newcommand\op{\star}
\newcommand\complexity[1]{\textsf{\upshape #1}}

\begin{document}
\maketitle

\section{Introduction}

\section{Preliminaries}

\section{Semantics}

\section{Results}

\begin{proposition}
  Let $S$ be any m-semiring. $S$ is idempotent if and only if $\ominus$
  is right-distributive over~$\oplus$ in $S$.
\end{proposition}

% This is wrong but why?
%\begin{proposition}
%   Let $K_1, K_2$ be two m-semirings and $h: K_1\to K_2$ a semiring
%   homomorphism. Then $h$ is an m-semiring homomorphism.
% \end{proposition}

\begin{proposition}
  Let $K_1, K_2$ be two m-semirings and $h: K_1\to K_2$ an m-semiring
  homomorphism. Then:
    \begin{compactenum}[(i)]
      \item If $\mathbb{K}_1$ is idempotent, then $\mathbb{K}_2$ is
        idempotent.
      \item If $\otimes$ is distributive over $\ominus$ in $K_1$, the same
        is true in~$\mathbb{K}_2$.
    \end{compactenum}
\end{proposition}

\begin{corollary}
  The following m-semirings are idempotent and have distributivity of
  $\otimes$ over $\ominus$: the temporal semiring, the Bool[X] semiring,
  the Which[X] semiring, etc.
\end{corollary}

\begin{theorem}
  For any binary query $q$, for any $C\in\mathbb{N}$, for $\op\in \{\leq,
  =,\geq\}$ consider the queries:
  \begin{align*}
    Q^{\op C}_1&=\Pi_{\#1}(\sigma_{\#2\op C}(\gamma_{\#1}[1:+](q(R))))\\
  Q^{\geq C}_2&=\Pi_{\#1}(q(R)\bowtie_{\#1=\#3\land\#2<\#4}
    q(R)\bowtie_{\#1=\#5\land\#4<\#6}\dots\bowtie_{\#1=\#(2C+1)\land\#(2C)<\#(2C+2)}q(R))\\
Q^{\leq C}_2&=q(R)-Q_2^{\geq C+1}\\
Q^{=C}_2&=Q_2^{\geq C} -Q_2^{\geq C+1}
\end{align*}
    Then:
    \begin{compactenum}[(i)]
      \item
        For any commutative m-semiring $\mathbb{K}$ which is
        idempotent and such that $\otimes$ is
        distributive over $\ominus$, for any
        $\mathbb{K}$-instance $\hat I$,
        $\langle Q^{\op C}_1\rangle^{\hat I}=\langle Q^{\op C}_2\rangle^{\hat I}$.
      \item There exists a commutative m-semiring $\mathbb{K}_1$ which is
        idempotent but does
        not have distributivity of $\otimes$ over $\ominus$, and a
        $\mathbb{K}_1$-instance $\hat I_1$ such that
        $\langle Q^{\op C}_1\rangle^{\hat I_1}\neq\langle Q^{\op C}_2\rangle^{\hat I_1}$.
      \item There exists a commutative m-semiring $\mathbb{K}_2$ that has
        distributivity of $\otimes$ over $\ominus$ but which is not
        idempotent, and a
        $\mathbb{K}_2$-instance $\hat I_2$ such that
        $\langle Q^{\op C}_1\rangle^{\hat I_2}\neq\langle Q^{\op C}_2\rangle^{\hat I_2}$.
    \end{compactenum}
\end{theorem}

\begin{proposition}
  For any binary query $q$, for any $C\in\mathbb{N}^*$, for $\op\in \{\leq,
  =,\geq\}$, for any commutative $m$-semiring $\mathbb{K}$,
  computing
  $\langle Q^{\op C}\rangle^{\hat I}$ for the query
\[    Q^{\op C}_1=\Pi_{\#1}(\sigma_{\#2\op C}(\gamma_{\#1}[1:+](q(R))))
\] over a $\mathbb{K}$-instance $\hat I$ is \complexity{\#P}-hard.
\end{proposition}

\section{Implementing in Practice}

\section{Experiments}

\section{Conclusion}

\end{document}
