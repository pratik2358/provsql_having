\documentclass{scrartcl}
\usepackage[left=1cm, right=1cm]{geometry}
\usepackage{amssymb,amsfonts,amsmath,amsthm}
\usepackage{paralist}
\usepackage{graphicx}
\usepackage{xcolor}
\usepackage[bb=boondox]{mathalfa}
\usepackage{algorithm}
\usepackage{algorithmic}
\usepackage{booktabs}
\usepackage[pdfborder={0 0 0}]{hyperref}
\usepackage{xcolor}

\usepackage{pifont}
\newcommand{\ok}{\text{\color{green!50!black}\ding{51}}}
\newcommand{\ko}{\text{\color{red!50!black}\ding{55}}}

\DeclareMathAlphabet{\mathdutchcal}{U}{dutchcal}{m}{n}
\SetMathAlphabet{\mathdutchcal}{bold}{U}{dutchcal}{b}{n}
\DeclareMathAlphabet{\mathdutchbcal}{U}{dutchcal}{b}{n}
\newcommand{\dq}[1]{\text{\enquote{#1}}}
\newcommand{\op}{\mathbin{\mathsf{\star}}}
\makeatletter
\newcommand{\bigplus}{%
  \DOTSB\mathop{\mathpalette\mattos@bigplus\relax}\slimits@
}
\newcommand\mattos@bigplus[2]{%
  \vcenter{\hbox{%
    \sbox\z@{$#1\sum$}%
    \resizebox{!}{0.9\dimexpr\ht\z@+\dp\z@}{\raisebox{\depth}{$\m@th#1+$}}%
  }}%
  \vphantom{\sum}%
}
\newcommand{\mline}[1]{%
  \begin{multiline}
    #1
  \end{multiline}
}
\newcommand{\llbracket}{[\![}
\newcommand{\rrbracket}{]\!]}
\title{On the Semantics of the Provenance of HAVING queries in Semirings
with Monus}

\renewcommand{\epsilon}{\varepsilon}
\renewcommand{\le}{\leqslant}
\renewcommand{\ge}{\geqslant}
\renewcommand{\leq}{\leqslant}
\renewcommand{\geq}{\geqslant}

\newtheorem{corollary}{Corollary}
\newtheorem{proposition}{Proposition}
\newtheorem{theorem}{Theorem}
\newtheorem*{remark}{Remark}
\theoremstyle{definition}
\newtheorem{definition}{Definition}
\newtheorem{lemma}{Lemma}

\newcommand\complexity[1]{\textsf{\upshape #1}}

\newcommand\lean[2]{\href{https://provsql.org/lean-docs/Provenance/#1.html\##2}{\fbox{\includegraphics[height=.6em]{lean}}}}

\DeclareMathAlphabet{\mathdutchcal}{U}{dutchcal}{m}{n}
\SetMathAlphabet{\mathdutchcal}{bold}{U}{dutchcal}{b}{n}
\DeclareMathAlphabet{\mathdutchbcal}{U}{dutchcal}{b}{n}

\newcommand{\pierre}[1]{{\color{magenta}[\textbf{Pierre:} #1]}}


\usepackage[appendix=append]{apxproof}
\begin{document}
\maketitle

\section{Introduction}

\section{Preliminaries}

\begin{table}
 \centering\begin{tabular}{rccc}
\toprule
&\bfseries idempotent&\bfseries absorptive&\bfseries $\otimes$ distr.\ over
$\ominus$\\
\midrule
$\mathbb{B}$ \lean{Semirings/Bool}{instSemiringWithMonusBool}&
\ok\ \lean{Semirings/Bool}{Bool.idempotent}&
\ok\ \lean{Semirings/Bool}{Bool.absorptive}&
\ok\ \lean{Semirings/Bool}{Bool.mul_sub_left_distributive}\\
$\mathbb{B}[X]$ \lean{Semirings/BoolFunc}{instSemiringWithMonusBoolFunc}&
\ok\ \lean{Semirings/BoolFunc}{BoolFunc.idempotent}&
\ok\ \lean{Semirings/BoolFunc}{BoolFunc.absorptive}&
\ok\ \lean{Semirings/BoolFunc}{BoolFunc.mul_sub_left_distributive}\\
$\mathrm{Why}[X]$ \lean{Semirings/Why}{instSemiringWithMonusWhy}&
\ok\ \lean{Semirings/Why}{Why.idempotent}&
\ko\ \lean{Semirings/Why}{Why.not_absorptive}&
\ok\ \phantom{\lean{Semirings/Why}{Why.mul_sub_left_distributive}}\\
$\mathrm{Which}[X]$ \lean{Semirings/Which}{instSemiringWithMonusWhich}&
\ok\ \lean{Semirings/Which}{Which.idempotent}&
\ko\ \lean{Semirings/Which}{Which.not_absorptive}&
\ok\ \phantom{\lean{Semirings/Which}{Which.mul_sub_left_distributive}}\\
$\mathrm{Temporal}[X]$ \phantom{\lean{Semirings/Temporal}{instSemiringWithMonusTemporal}}&
\ok\ \phantom{\lean{Semirings/Temporal}{Temporal.idempotent}}&
\ok\ \phantom{\lean{Semirings/Temporal}{Temporal.absorptive}}&
\ok\ \phantom{\lean{Semirings/Why}{Why.mul_sub_left_distributive}}\\
$\mathbb{T}$ \lean{Semirings/Tropical}{instSemiringWithMonusTropicalWithTopReal}&
\ok\ \lean{Semirings/Tropical}{Tropical.idempotent}&
\ok\ \lean{Semirings/Tropical}{TropicalN.absorptive}&
\ok\ \lean{Semirings/Tropical}{TropicalN.mul_sub_left_distributive}\\
$\mathrm{TVL}, \mathbb{S}$ \lean{Semirings/MinMax}{instSemiringWithMonusMaxMinTVL}&
\ok\ \lean{Semirings/MinMax}{MinMax.idempotent}&
\ok\ \lean{Semirings/MinMax}{MinMax.absorptive}&
\ko\ \lean{Semirings/MinMax}{TVL.not_mul_sub_left_distributive}\\
$\mathbb{N}$ \lean{Semirings/Nat}{instSemiringWithMonusNat}&
\ko\ \lean{Semirings/Nat}{Nat.not_idempotent}&
\ko\ \lean{Semirings/Nat}{Nat.not_absorptive}&
\ok\ \lean{Semirings/Nat}{Nat.mul_sub_left_distributive}\\
$\mathbb{N}[X]$ \lean{Semirings/How}{instSemiringWithMonusMvPolynomialNat}&
\ko\ \lean{Semirings/How}{How.not_idempotent}&
\ko\ \lean{Semirings/How}{How.not_absorptive}&
\ok\ \phantom{\lean{Semirings/How}{How.mul_sub_left_distributive}}\\
$\textrm{\L ukasiewicz}$ \lean{Semirings/Lukasiewicz}{instSemiringWithMonusMvPolynomialNat}&
\ok\ \lean{Semirings/Lukasiewicz}{Lukasiewicz.idempotent}&
\ok\ \lean{Semirings/Lukasiewicz}{Lukasiewicz.absorptive}&
\ok\ \phantom{\lean{Semirings/Lukasiewicz}{Lukasiewicz.mul_sub_left_distributive}}\\
\bottomrule
\end{tabular}
\caption{Properties of common (commutative) m-semirings}
\label{tab:semirings}
\end{table}


\section{Semantics}
Extending on the algebra over annotated
relations presented in ***  by
adding a new predicate form that allows
comparison between aggregations we fill the semantics gap that
was noted and left for future work.
\paragraph{Setup.}
Let $(K,\oplus,\otimes,\mathbb{0}_K,\mathbb{1}_K,\delta,\ominus)$ be an m-semiring of annotations with an
added $\delta$ operation and, $(M,+_M,\mathbb{0}_M)$ be an
additive monoid of aggregate values. Let $*:K\times M \mapsto K$ be a scalar action of $M$ on $K$, such that for $a \in K, b \in M$,
$a * b \in K$.
We fix a finite set $\mathbf{X}$ of variables. Each $\mathbf{x}\in\mathbf{X}$ have provenance annotations in $K$ and values in $M$
given by a semimodule ($K\times M$, say $\mathcal{M}$) map
\[
\mathfrak{Z}:\mathbf{X}\rightarrow \mathcal{M}\; \text{and} \quad \mathfrak{Z}^{(K)}(\mathbf{x})=\mathbf{x}^{(K)}, \; \mathfrak{Z}^{(M)}(\mathbf{x})=\mathbf{x}^{(M)}
\]
An aggregation $\mathdutchcal{a}$ is defined by an indicator function
$I_\mathdutchcal{a}:\mathbf{X}\rightarrow \{0,1\}$
on variables from $\mathbf{X}$ that participate in the aggregation as
\[
\mathdutchcal{a} = {\bigoplus_\mathcal{M}}_{\mathbf{x}\in \mathrm{supp}(\mathdutchcal{a})}(\mathbf{x}^{(K)},\mathbf{x}^{(M)})
\]
where,
$\mathrm{supp}(\mathdutchcal{a})=\{\,\mathbf{x} \mid I_\mathdutchcal{a}(\mathbf{x})=1 \,\}$ and
$\bigoplus_{\mathcal{M}}$ is a formal $\mathcal{M}-$sum over
variables $\mathbf{x}$ involved in the aggregation.
A possible world is represented by a subset $W\subseteq\mathbf{X}$.

\pierre{What precedes need to be made more formal (e.g., there should be
  a formal definition of a semiring with~$\delta$, and of a semimodule,
  probably in the previous section; this
  is important, for example for now it is not really explained how an
  aggregation is defined -- a formal sum is not an element of a
semimodule; we should talk about semimodule elements of a given specific
form, or have a normal form for semimodule elements.)}

\begin{definition}[Possible-World Semantics]\label{def:pw_semantics}
  \pierre{Every definition needs to be self-contained, and not depend on
    external variables; so we should say something like ``For any
    m-semiring $K$, for any additive monoid $M$, for any finite set of variables
  $\mathbf{X}$\dots'''}

For any two aggregations $\mathdutchcal{u},\mathdutchcal{v}$, we define the comparison operator $\oslash$ at the semimodule level as,
\[
[\mathdutchcal{u} \oslash \mathdutchcal{v}]_{\mathit{op}\in\{ =, \neq , >, <, \geq, \leq\}}:\mathcal{M} \times \mathcal{M} \rightarrow K
\]
as:
\begin{multline*}
[\mathdutchcal{u} \oslash \mathdutchcal{v}]_{\op}=
\bigoplus_{W\subseteq\mathbf{X}}
\Biggl(
  \Bigl( \bigotimes_{W\cap\mathrm{supp}(\mathdutchcal{u})} \mathbf{x}^{(K)} \Bigr) \;\otimes\; \Bigl( \bigotimes_{W\cap\mathrm{supp}(\mathdutchcal{v})} \mathbf{x}^{(K)} \Bigr)
\otimes
  \chi_{\mathrm{op}}\bigl(  {\bigplus_M}_{W\cap\mathrm{supp}(\mathdutchcal{u})} \mathbf{x}^{(M)}, {\bigplus_M}_{W\cap\mathrm{supp}(\mathdutchcal{v})} \mathbf{x}^{(M)}   \bigr)
\;\otimes\;\\
  \Biggl(\mathbb{1}_K \ominus \Biggl( \bigoplus_{(\mathbf{X}\setminus W)\cap\mathrm{supp}(\mathdutchcal{u})} \mathbf{x}^{(K)} \;\oplus\; \bigoplus_{(\mathbf{X}\setminus W)\cap\mathrm{supp}(\mathdutchcal{u})} \mathbf{x}^{(K)} \Biggr)\Biggr)
\Biggr)
\end{multline*}
where, for any two elements $a, b\in M$,
$\chi_{\op}(a,b)
=
\begin{cases}
\mathbb{1}_K & \text{if } a\op b\\
\mathbb{0}_K & \text{otherwise.}
\end{cases}$
\end{definition}

\pierre{We need to use these definitions to actually define a semantics
for the $\sigma$ operator when one of the column is the result of an
aggregation.}

\section{Results}
We reason on the generality and correctness of the possible-world semantics we proposed for aggregate comparison predicates, by showing that our semantics agrees with that of standard JOIN equivalent queries for simple HAVING-COUNT queries. Concretely we show that if we restrict our domain of provenance annotations to a commutative $m-$semiring with two other algebraic constraints: additive idempotence and distributivity of $\otimes$ over $\ominus$, provenance evaluation using our possible-world semantics is equivalent to the provenance evaluated for the JOIN-based rewritings of simple HAVING-COUNT queries.

\begin{theorem}\label{th:correctness}
  For any binary query $q$, for any $C\in\mathbb{N}_{\ge1}$, for $\op\in \{\leq,
  =,\geq\}$ consider the queries:
\begin{align*}
    Q^{\op C}_1&=\Pi_{\#1}(\sigma_{\#2\op C}(\gamma_{\#1}[1:+](q)))\\
    Q^{\geq C}_2&=\begin{cases}\Pi_{\#1}(q) & \text{if } C=1,\\[1mm]\Pi_{\#1}(q\bowtie_{\#1=\#3\land\#2<\#4}
    q\bowtie_{\#1=\#5\land\#4<\#6}\dots\bowtie_{\#1=\#(2C-1)\land\#(2C-2)<\#(2C)}q) & \text{if }C\ge 2\end{cases}\\
    Q^{\leq C}_2&= Q^{\geq 1}_2-Q_2^{\geq C+1}\\
Q^{=C}_2&=Q_2^{\geq C} -Q_2^{\geq C+1}
\end{align*}
    Then:
    \begin{compactenum}[(i)]
      \item
        For any commutative m-semiring $\mathbb{K}$ which is
        idempotent and such that $\otimes$ is
        distributive over $\ominus$, for any
        $\mathbb{K}$-instance $\hat I$,
        $\langle Q^{\op C}_1\rangle^{\hat I}=\langle Q^{\op C}_2\rangle^{\hat I}$.
      \item There exists a commutative m-semiring $\mathbb{K}_1$ which is
        idempotent but does
        not have distributivity of $\otimes$ over $\ominus$, and a
        $\mathbb{K}_1$-instance $\hat I_1$ such that
        $\langle Q^{\op C}_1\rangle^{\hat I_1}\neq\langle Q^{\op C}_2\rangle^{\hat I_1}$.
      \item There exists a commutative m-semiring $\mathbb{K}_2$ that has
        distributivity of $\otimes$ over $\ominus$ but which is not
        idempotent, and a
        $\mathbb{K}_2$-instance $\hat I_2$ such that
        $\langle Q^{\op C}_1\rangle^{\hat I_2}\neq\langle Q^{\op C}_2\rangle^{\hat I_2}$.
    \end{compactenum}
\end{theorem}
\begin{proof}[Proof sketch]
The two queries differ only in how they implement the \textsc{having} condition. So we may fix one group key value $a$ and compare the single provenance expression produced for that group; equality for every $a$ then implies equality of the full query results.

  For simplicity, let us assume for now that $q$ has no duplicate tuple.For this fixed $a$, let $U$ be the finite collection of tuples of $q$ with first attribute $a$, with annotations $\alpha_u\in\mathbb K$. The query $Q^{\ge C}_2$ with strict self-join ordering predicate is designed to ``exhibit'' $C$ distinct tuples in the group: a satisfying valuation of these \textsc{join} conditions selects $C$ tuples (arranged by the strict inequalities in the join predicate), and the semantics contributes the product of their annotations. Summing over all satisfying valuations therefore yields the necessary provenance: the $\oplus$-sum of all products $A_W=\bigotimes_{u\in W}\alpha_u$ over $C$-element choices $W\subseteq U$. Call this expression $S_C(U)$.

The possible-world semantics of the \textsc{having} predicate looks quite different at first: it sums over all subsets $W\subseteq U$ (possible worlds), keeps those with $|W|\ge C$, and multiplies by a factor enforcing that tuples outside $W$ are ``absent.'' Using distributivity of $\otimes$ over $\ominus$ and $\oplus$, the contribution associated with each $W$ can be rewritten in a normal form
\[
T_U(W)=A_W\ominus\bigoplus_{u\in U\setminus W}A_{W\cup\{u\}},
\]
so that the provenance becomes a single aggregated expression
\[
F_C(U)=\bigoplus_{|W|\ge C}T_U(W).
\]
Intuitively, $T_U(W)$ takes the monomial for choosing exactly the tuples in $W$ and removes away all one-step extensions of $W$, which is precisely what the ``absence'' factor is meant to enforce.

The crux is that, although $F_C(U)$ and $S_C(U)$ arise from very different-looking constructions, if we pick any $u\in U$, write $U'=U\setminus\{u\}$, and $\beta=\alpha_u$, then both expressions split into two disjoint contributions: those worlds that do not use $u$, and those that do use $u$. This yields the same include/exclude recurrence for both  $F_C(U)$ and $S_C(U)$  (for all $C\ge 1$):
\[
\square_C(U)=\square_C(U')\ \oplus\ \bigl(\square_{C-1}(U')\otimes\beta\bigr).
\]
For $S_C$ this is easy to obtain from partitioning $C$-subsets by membership of $u$. For $F_C$, establishing the same recurrence is the interesting part: it relies residual characterization of $\ominus$ together with an $W$-expansion lemma ensuring that products $A_V$ are bounded by suitable $\oplus$-sums of the terms $T_U(W)$. Once both recurrences are in place and the base cases match, induction on $|U|$ gives $F_C(U)=S_C(U)$, proving the $\ge$ case.

  Finally, the cases $\op\in\{=,\le\}$ are reduced to the already established $\ge$ case by expressing the corresponding predicates as differences of $\ge$-predicates and lifting these boolean identities through the semantics using distributivity of $\otimes$ over $\ominus$ and right-distributivity of $\ominus$ over $\oplus$ (which follows from idempotence). Since $a$ was arbitrary, the provenance evaluated for both the queries agree for every output tuple. We show a full proof for this theorem in detail in Section~\ref{sec:proofoftheorem}; we also discuss there how the argument extends to multiset semantics.  \textcolor{red}{to do: write proofs for ii and iii}
\end{proof}
\begin{proposition}
  For any binary query $q$, for any $C\in\mathbb{N}^*$, for $\op\in \{\leq,
  =,\geq\}$, for any commutative $m$-semiring $\mathbb{K}$,
  computing
  $\langle Q^{\op C}\rangle^{\hat I}$ for the query
\[    Q^{\op C}_1=\Pi_{\#1}(\sigma_{\#2\op C}(\gamma_{\#1}[1:+](q(R))))
\] over a $\mathbb{K}$-instance $\hat I$ is \complexity{\#P}-hard.
\end{proposition}
\textcolor{red}{Cannot say this for any $\mathbb{K}$. Counter example $\mathbb{B}$}
\section{Proof of Theorem~\ref{th:correctness}}\label{sec:proofoftheorem}
\pierre{The first few results are really properties of m-semirings, we
should put them in the preliminaries.}
We start by proving simple results that we would later need in the proof. We first show that in an $m$-semiring, additive idempotence is equivalent to $\ominus$ being right-distributive over $\oplus$.
\begin{proposition}
  \lean{SemiringWithMonus}{idempotent_iff_add_monus}
  Let $S$ be any m-semiring. $S$ is idempotent if and only if $\ominus$
  is right-distributive over~$\oplus$ in $S$.
\end{proposition}
\begin{proof}
  \textbf{($\Rightarrow$)}
  \lean{SemiringWithMonus}{add_monus_of_idempotent}
  Assume S to be additively idempotent. Define a preorder $\leq$ such that,
  if $a\leq b$ then $\exists x,\; a\oplus x=b$.
  So, $a \oplus b = a \oplus (a \oplus x)=(a\oplus a)\oplus x=a \oplus x = b$. \\$\therefore a\leq b \implies a\oplus b=b$.
  \\Conversely, if $a\oplus b = b$, choose $x=b$. And trivially, $a\oplus x =a\oplus b=b$
  $\implies a \leq b$.
  \\$\therefore a\leq b \iff a \oplus b =b$. Therefore, $(S,\leq)$ is indeed a partial order and $\oplus$ is monotone under $\leq$.
  \\Now, $\forall u, a \leq u$ and $b \leq u$
  \\ $a \oplus u = u$ and $b\oplus u =u$.
  \\But, $a\leq a \oplus b$ and $b \leq a \oplus b$
  \\$\therefore a \oplus b $ is the least such $u$ for which $a\leq u $ and $b \leq u$.
  So, $\oplus$ is the join of a semilattice of $(S,\leq)$.

  \paragraph{i.}
  By definition of monus : $x \ominus y \leq z \iff x \leq y \oplus z$,
  $a\leq a\oplus b\leq c \oplus ((a\oplus b)\ominus c) \implies a\ominus c\leq (a\oplus b)\ominus c$.
  Similarly, $b\ominus c\leq (a\oplus b)\ominus c$.
  \\$\therefore (a\ominus c)\oplus (b\ominus c) \leq (a\oplus b) \ominus c$
  \paragraph{ii.} We have $a\leq c \oplus (a\ominus c)$ and $b\leq c\oplus (b\ominus c)$. So, $a\oplus b \leq c\oplus c \oplus (a\ominus c) \oplus (b\ominus c)$.
  By idempotence of $\oplus$,  $a\oplus b \leq c \oplus (a\ominus c) \oplus (b\ominus c)$.
\\$\therefore $ $(a\oplus b) \ominus c \leq (a\ominus c)\oplus (b\ominus c) $, by definition of monus.
 \\ From \textbf{i.} \& \textbf{ii.}
  $$\boxed{(a\oplus b) \ominus c = (a\ominus c)\oplus (b\ominus c)}$$
\textbf{($\Leftarrow$)}
  \lean{SemiringWithMonus}{idempotent_of_add_monus}
Assume that $\ominus$ of S is right-distributive:
\[
  (a\oplus b)\ominus c
  = (a\ominus c)\oplus(b\ominus c)\;\forall a,b,c\in S.
\]

For any arbitrary $a\in S$ substitute $b=a$ and $c=a$:
\[
  (a\oplus a)\ominus a
  = (a\ominus a)\oplus(a\ominus a).
\]
Since $a\ominus a=0$, the right-hand side is $0$. And $(a\oplus a)\ominus a = 0\implies a\oplus a\leq a$ by definition of monus.
  But, trivially $a\le a\oplus a$, so antisymmetry of $(S,\leq)$ gives $a\oplus a=a$.
  Thus, $$\boxed{a \oplus a=a, \; \forall a\in S}$$
\smallskip
Both sides, prove the equivalence.
\end{proof}
% This is wrong but why?
%\begin{proposition}
%   Let $K_1, K_2$ be two m-semirings and $h: K_1\to K_2$ a semiring
%   homomorphism. Then $h$ is an m-semiring homomorphism.
% \end{proposition}
\begin{corollary}[Right-distributivity of $\ominus$ over finite $\oplus$-sums]\label{lem:finite-add-monus}
Let \((K,\oplus,\otimes,\mathbb{0},\mathbb{1},\delta,\ominus)\) be an
m-semiring in which \(\ominus\) is right-distributive over \(\oplus\), i.e.,
\[
(a \oplus b) \ominus c = (a \ominus c) \oplus (b \ominus c)
\quad\text{for all } a,b,c \in K.
\]
Then for every \(n \ge 1\) and all \(a_1,\dots,a_n,c \in K\),
\[
\Bigl(\bigoplus_{i=1}^n a_i\Bigr)\ominus c
=
\bigoplus_{i=1}^n (a_i \ominus c).
\]
\end{corollary}

\begin{proof}
We proceed by induction on \(n \ge 1\).

For \(n=1\), the identity is trivial.
For \(n=2\), the statement is exactly the assumed right-distributivity law:
\[
(a_1 \oplus a_2)\ominus c = (a_1\ominus c)\oplus(a_2\ominus c).
\]

Assume the identity holds for \(n-1\).
Then
\[
\begin{aligned}
\Bigl(\bigoplus_{i=1}^n a_i\Bigr)\ominus c
&= \Bigl(\bigl(\bigoplus_{i=1}^{n-1} a_i\bigr)\oplus a_n\Bigr)\ominus c \\
&= \Bigl(\bigoplus_{i=1}^{n-1} a_i\Bigr)\ominus c \;\oplus\; (a_n\ominus c) \\
&= \bigoplus_{i=1}^{n-1} (a_i\ominus c) \;\oplus\; (a_n\ominus c)
= \bigoplus_{i=1}^n (a_i\ominus c),
\end{aligned}
\]
where the second equality uses right-distributivity and the third uses the
induction hypothesis.
\end{proof}
We now isolate two elementary inequalities on the residual charecterization of $\ominus$ and the natural order defined for commutativ $m-$semirings.
\begin{proposition}
  Let $\mathbb{K}_1, \mathbb{K}_2$ be two m-semirings and $h:
  \mathbb{K}_1\to \mathbb{K}_2$ an \emph{injective} m-semiring
  homomorphism. Then:
    \begin{compactenum}[(i)]
    \item
      \lean{SemiringWithMonus}{idempotent_of_injective_homomorphism_idempotent}
      If $\mathbb{K}_2$ is idempotent, then $\mathbb{K}_1$ is
        idempotent.
      \item
        \lean{SemiringWithMonus}{mul_sub_left_of_injective_homomorphism_mul_sub_left}
        If $\otimes$ is left-distributive over $\ominus$ in $\mathbb{K}_2$, the same
        is true in~$\mathbb{K}_1$.
    \end{compactenum}
\end{proposition}


\begin{corollary}
  The following m-semirings are idempotent and have distributivity of
  $\otimes$ over $\ominus$: the temporal semiring, the Bool[X] semiring,
  the Which[X] semiring, etc.
\end{corollary}
\begin{lemma}\label{lem:x-y<=x}
  \pierre{We should directly refer to the definition of an m-semiring.}
Let $(K,\oplus,\mathbb 0)$ be an idempotent commutative monoid, equipped with an
operation $\ominus$ satisfying the residuation law: for all $x,y,z\in K$,
\begin{equation}\label{eq:residuation-monus}
  x\ominus y \le z \quad\Longleftrightarrow\quad x \le y\oplus z,
\end{equation}
where $\le$ is the natural order $a\le b \iff a\oplus b=b$.
Then for all $x,y\in K$,
\[
  x\ominus y \le x.
\]
\end{lemma}

\begin{proof}
By \eqref{eq:residuation-monus} with $z=x$, we have
\[
x\ominus y \le x \quad\Longleftrightarrow\quad x \le y\oplus x.
\]
But $x\le y\oplus x$ holds because $x\oplus(y\oplus x)=(x\oplus x)\oplus y=x\oplus y=y\oplus x$
(using associativity, commutativity and idempotence of $\oplus$).
Hence $x\ominus y\le x$.
\end{proof}

  \begin{lemma}\label{lem:x<=y+(x-y)}
    \pierre{Same.}
  Under the assumptions of Lemma~\ref{lem:x-y<=x}, for all $x,y\in K$,
\[
  x \le y\oplus (x\ominus y).
\]
\end{lemma}

\begin{proof}
Apply \eqref{eq:residuation-monus} with $z=x\ominus y$:
since $x\ominus y\le x\ominus y$ holds trivially, \eqref{eq:residuation-monus} yields
\[
x \le y\oplus (x\ominus y),
\]
as required.
\end{proof}
Next, we state a bounding lemma that we later use in the proof to show that summing $T_U(W)$ over all $W\supseteq V$ always dominates the product $A_V$ for $V\subseteq U$.
\begin{lemma}\label{lem:upward-expansion}
Let $(K,\oplus,\otimes,\mathbb 0,\mathbb 1,\ominus)$ be an idempotent $m$-semiring,
and write the natural order as $a\le b \iff a\oplus b=b$.
Assume that $\ominus$ is the (right) residual of $\oplus$, i.e.\ for all $x,y,z\in K$,
\begin{equation}\label{eq:residuation}
  x\ominus y \le z \quad\Longleftrightarrow\quad x \le y\oplus z .
\end{equation}
  Fix a finite set $U$ and elements $(\alpha)_{x\in U}\in K$.
For $W\subseteq U$ define
\[
  A_W := \bigotimes_{x\in W}\alpha_x ,
  \qquad
  T_U(W) := A_W \ominus \bigoplus_{x\in U\setminus W} A_{W\cup\{x\}} .
\]
Then for every $V\subseteq U$,
\begin{equation}\label{eq:upward-expansion}
  A_V \;\le\; \bigoplus_{\substack{W:\;V\subseteq W\subseteq U}} T_U(W).
\end{equation}
\end{lemma}

\begin{proof}
We prove \eqref{eq:upward-expansion} by induction on $d:=|U\setminus V|$.

\smallskip\noindent
\textbf{Base case ($d=0$).}
Then $V=U$. Since $U\setminus U=\varnothing$ and $\bigoplus_{\varnothing}(\cdot)=\mathbb 0$,
\[
  T_U(U)=A_U\ominus \mathbb 0.
\]
We claim $A_U\ominus\mathbb 0=A_U$. Indeed, by \eqref{eq:residuation}, for any $z\in K$,
\[
  A_U\ominus\mathbb 0 \le z \iff A_U \le \mathbb 0\oplus z = z,
\]
so $A_U\ominus\mathbb 0$ is the greatest element below $A_U$, hence equals $A_U$.
Therefore,
\[
  \bigoplus_{\substack{W:\;U\subseteq W\subseteq U}} T_U(W)=T_U(U)=A_U=A_V,
\]
and \eqref{eq:upward-expansion} holds.

\smallskip\noindent
\textbf{Inductive step.}
Assume $d>0$ and that \eqref{eq:upward-expansion} holds for all pairs $(U,V')$
with $V\subsetneq V'\subseteq U$ and $|U\setminus V'|<d$.
Let
\[
  Y := \bigoplus_{x\in U\setminus V} A_{V\cup\{x\}} \in K.
\]
From Lemma~\ref{lem:x<=y+(x-y)} we obtain the standard inequality
\begin{equation}\label{eq:M2-lemma1}
  X \le Y \oplus (X\ominus Y)\qquad\text{for all }X,Y\in K,
\end{equation}
by instantiating \eqref{eq:residuation} with $z=X\ominus Y$ (since $X\ominus Y\le X\ominus Y$).
Applying \eqref{eq:M2-lemma1} to $X=A_V$ gives
\[
  A_V \le Y \oplus (A_V\ominus Y).
\]
By definition of $T_U(V)$ we have $A_V\ominus Y = T_U(V)$, hence
\begin{equation}\label{eq:AV-bound}
  A_V \le T_U(V)\ \oplus\ \bigoplus_{x\in U\setminus V} A_{V\cup\{x\}}.
\end{equation}

For each $x\in U\setminus V$ set $V_x:=V\cup\{x\}$. Then $|U\setminus V_x|=d-1$,
so the induction hypothesis yields
\[
  A_{V_x}\le \bigoplus_{\substack{W:\;V_x\subseteq W\subseteq U}} T_U(W).
\]
Taking the $\oplus$-sum over all $x\in U\setminus V$ and using monotonicity of $\oplus$,
\[
  \bigoplus_{x\in U\setminus V} A_{V\cup\{x\}}
  =
  \bigoplus_{x\in U\setminus V} A_{V_x}
  \le
  \bigoplus_{x\in U\setminus V}\ \bigoplus_{\substack{W:\;V_x\subseteq W\subseteq U}} T_U(W).
\]
Every $W$ appearing on the right satisfies $V\subseteq V_x\subseteq W\subseteq U$, hence $V\subseteq W\subseteq U$.
Therefore, using associativity/commutativity/idempotence of $\oplus$ to ignore duplicates and reorder summands,
\[
  \bigoplus_{x\in U\setminus V}\ \bigoplus_{\substack{W:\;V_x\subseteq W\subseteq U}} T_U(W)
  \le
  \bigoplus_{\substack{W:\;V\subseteq W\subseteq U}} T_U(W).
\]
Combining with \eqref{eq:AV-bound} yields
\[
  A_V \le T_U(V)\ \oplus\ \bigoplus_{\substack{W:\;V\subseteq W\subseteq U}} T_U(W)
  =
  \bigoplus_{\substack{W:\;V\subseteq W\subseteq U}} T_U(W),
\]
since the right-hand $\oplus$-sum already includes the summand $T_U(V)$ (take $W=V$).
This proves \eqref{eq:upward-expansion} for $V$ and completes the induction.
\end{proof}
Next, we show the recurrences for $F_C$ and $S_C$.
\begin{lemma}[Recurrence for $F_C$]\label{lem:FC_recurrence}
Let $(K,\oplus,\otimes,\mathbb 0,\mathbb 1,\ominus)$ be an idempotent $m$-semiring.
Write the natural order as $a\le b \iff a\oplus b=b$, and assume $\ominus$ satisfies
residuation $x\ominus y\le z \iff x\le y\oplus z$.
Assume moreover that $\otimes$ distributes over $\oplus$ and over $\ominus$, i.e.
\[
r\otimes(a\oplus b)=(r\otimes a)\oplus(r\otimes b),
\qquad
r\otimes(a\ominus b)=(r\otimes a)\ominus(r\otimes b).
\]
Fix a finite set $U$ and elements $(\alpha)_{x\in U}\in K$.
For $W\subseteq U$ define
\[
A_W:=\bigotimes_{x\in W}\alpha_x,
\qquad
T_U(W):=A_W\ominus \bigoplus_{x\in U\setminus W}A_{W\cup\{x\}},
\]
and for $C\in\mathbb N$ define
\[
F_C(U):=\bigoplus_{\substack{W\subseteq U\\|W|\ge C}} T_U(W).
\]
Let $u\in U$, put $U':=U\setminus\{u\}$ and $\beta:=\alpha_u$.
Then for all $C\in\mathbb N$ with $C\ge 1$,
\[
F_C(U)=F_C(U')\ \oplus\ \bigl(F_{C-1}(U')\otimes\beta\bigr).
\]
\end{lemma}

\begin{proof}
Fix $u\in U$, set $U':=U\setminus\{u\}$ and $\beta:=\alpha_u$, and let $C\ge 1$.
Split  $F_C(U)$ according to whether $u\in W$:
\begin{equation}\label{eq:split-I-II}
F_C(U)
=
\underbrace{\bigoplus_{\substack{W\subseteq U'\\|W|\ge C}}T_U(W)}_{(I).\;u\notin W}
\ \oplus\
\underbrace{\bigoplus_{\substack{W'\subseteq U'\\|W'|\ge C-1}}T_U(W'\cup\{u\})}_{(II).\;u\in W}.
\end{equation}

\smallskip\noindent\emph{Simplify $(II)$.}
Fix $W'\subseteq U'$. Then $A_{W'\cup\{u\}}=A_{W'}\otimes\beta$, and
$U\setminus(W'\cup\{u\})=U'\setminus W'$. Hence, using distributivity of $\otimes$ over $\oplus$,
\[
\bigoplus_{x\in U\setminus(W'\cup\{u\})}A_{W'\cup\{u,x\}}
=
\bigoplus_{x\in U'\setminus W'}\bigl(A_{W'\cup\{x\}}\otimes\beta\bigr)
=
\Bigl(\bigoplus_{x\in U'\setminus W'}A_{W'\cup\{x\}}\Bigr)\otimes\beta.
\]
Therefore, using distributivity of $\otimes$ over $\ominus$,
\[
T_U(W'\cup\{u\})
=
(A_{W'}\otimes\beta)\ominus
\Bigl(\bigoplus_{x\in U'\setminus W'}A_{W'\cup\{x\}}\Bigr)\otimes\beta
=
\Bigl(A_{W'}\ominus\bigoplus_{x\in U'\setminus W'}A_{W'\cup\{x\}}\Bigr)\otimes\beta
=
T_{U'}(W')\otimes\beta.
\]
Taking $\oplus$ over all $W'\subseteq U'$ with $|W'|\ge C-1$ and factoring $\beta$ using
distributivity of $\otimes$ over $\oplus$ yields
\begin{equation}\label{eq:II-simplifies}
(II)=\Bigl(\bigoplus_{\substack{W'\subseteq U'\\|W'|\ge C-1}}T_{U'}(W')\Bigr)\otimes\beta
=F_{C-1}(U')\otimes\beta.
\end{equation}

\smallskip\noindent\emph{Relate $T_U(W)$ and $T_{U'}(W)$ for $W\subseteq U'$.}
Fix $W\subseteq U'$. Since $U\setminus W=(U'\setminus W)\cup\{u\}$, we have
\[
\bigoplus_{x\in U\setminus W}A_{W\cup\{x\}}
=
\Bigl(\bigoplus_{x\in U'\setminus W}A_{W\cup\{x\}}\Bigr)\oplus (A_W\otimes\beta).
\]
Using
$x\ominus(y\oplus z)=(x\ominus y)\ominus z$, we get
\begin{equation}\label{eq:TU-rewrite}
T_U(W)
=
A_W\ominus\Bigl(\bigoplus_{x\in U'\setminus W}A_{W\cup\{x\}}\oplus(A_W\otimes\beta)\Bigr)
=
T_{U'}(W)\ominus(A_W\otimes\beta).
\end{equation}

\smallskip\noindent\emph{Show $F_C(U)\le F_C(U')\oplus(F_{C-1}(U')\otimes\beta)$.}
  By \eqref{eq:TU-rewrite} and the inequality $x\ominus y\le x$ (shown in Lemma~\ref{lem:x-y<=x}),
we have $T_U(W)\le T_{U'}(W)$ for all $W\subseteq U'$.
Taking $\oplus$ over all $W\subseteq U'$ with $|W|\ge C$ yields $(I)\le F_C(U')$.
Together with \eqref{eq:split-I-II} and \eqref{eq:II-simplifies}, this gives
\begin{equation}\label{eq:leq-direction}
F_C(U)\le F_C(U')\oplus(F_{C-1}(U')\otimes\beta).
\end{equation}

\smallskip\noindent\emph{Show $F_C(U')\le F_C(U)$.}
Fix $W\subseteq U'$ with $|W|\ge C$.
By the inequality $x\le y\oplus(x\ominus y)$ (shown in Lemma~\ref{lem:x<=y+(x-y)}) applied to
$x=T_{U'}(W)$ and $y=A_W\otimes\beta$, and using \eqref{eq:TU-rewrite}, we get
\begin{equation}\label{eq:TUprime-bound}
T_{U'}(W)\le (A_W\otimes\beta)\oplus T_U(W).
\end{equation}
Since $|W|\ge C$, the term $T_U(W)$ is one of the summands of $F_C(U)$, hence $T_U(W)\le F_C(U)$.
Moreover $A_W\otimes\beta=A_{W\cup\{u\}}$, and $|W\cup\{u\}|\ge C$.
By Lemma~\ref{lem:upward-expansion} applied to $V=W\cup\{u\}$ we have
\[
A_{W\cup\{u\}}
\le
\bigoplus_{\substack{Y:\;W\cup\{u\}\subseteq Y\subseteq U}} T_U(Y)
\le
\bigoplus_{\substack{Y\subseteq U\\|Y|\ge C}}T_U(Y)
=
F_C(U),
\]
so $A_W\otimes\beta\le F_C(U)$.
Plugging these two bounds into \eqref{eq:TUprime-bound} yields $T_{U'}(W)\le F_C(U)$.
Taking $\oplus$ over all $W\subseteq U'$ with $|W|\ge C$ gives $F_C(U')\le F_C(U)$.

\smallskip\noindent\emph{The recurrence.}
From \eqref{eq:leq-direction}, $F_C(U')\le F_C(U)$ and
$F_{C-1}(U')\otimes\beta=(II)\le F_C(U)$, we obtain
$F_C(U')\oplus(F_{C-1}(U')\otimes\beta)\le F_C(U)$.
Together with \eqref{eq:leq-direction}, this yields
\[
F_C(U)=F_C(U')\oplus(F_{C-1}(U')\otimes\beta),
\]
as required.
\end{proof}
\begin{lemma}[Recurrence for $S_C$]\label{lem:SC_recurrence}
  Let $(K,\oplus,\otimes,\mathbb 0,\mathbb 1,\ominus)$ be an idempotent $m$-semiring such that $\otimes$ distributes over $\oplus$:
\[
r\otimes(a\oplus b)=(r\otimes a)\oplus(r\otimes b)\qquad(\forall r,a,b\in K).
\]
Fix a finite set $U$ and a family $(\alpha)_{x\in U}\in K$.
For $W\subseteq U$ define
\[
A_W:=\bigotimes_{x\in W}\alpha_x
\quad\text{(with the convention }A_\varnothing:=\mathbb 1\text{),}
\]
and for $C\in\mathbb N$ define
\[
S_C(U):=\bigoplus_{\substack{W\subseteq U\\|W|=C}} A_W
\quad
(\text{with the convention that an empty $\oplus$-sum equals }\mathbb 0).
\]
Let $u\in U$, put $U':=U\setminus\{u\}$ and $\beta:=\alpha_u$.
Then for every $C\ge 1$,
\[
S_C(U)=S_C(U')\ \oplus\ \bigl(S_{C-1}(U')\otimes\beta\bigr).
\]
\end{lemma}

\begin{proof}
Fix $C\ge 1$ and $u\in U$, and write $U=U'\cup\{u\}$ with $U'=U\setminus\{u\}$.

Define two index families:
\[
\mathcal A:=\{\,W\subseteq U \mid |W|=C,\ u\notin W\,\},
\qquad
\mathcal B:=\{\,W\subseteq U \mid |W|=C,\ u\in W\,\}.
\]
Then $\mathcal A$ and $\mathcal B$ are disjoint and their union is the full family
\[
\mathcal A\ \dot\cup\ \mathcal B=\{\,W\subseteq U\mid |W|=C\,\},
\]
because every $W\subseteq U$ either contains $u$ or does not.

Therefore, by associativity and commutativity of $\oplus$ (finite sum over a disjoint
union equals the $\oplus$ of the two partial sums),
\begin{equation}\label{eq:split-SC}
S_C(U)
=
\bigoplus_{W\in\mathcal A} A_W\ \oplus\ \bigoplus_{W\in\mathcal B} A_W.
\end{equation}

\smallskip\noindent
\emph{(i) The $\mathcal A$-part.}
If $W\in\mathcal A$, then $u\notin W$ and $W\subseteq U'=U\setminus\{u\}$.
Conversely, if $W\subseteq U'$ and $|W|=C$, then $W\subseteq U$ and $u\notin W$.
Hence
\[
\mathcal A=\{\,W\subseteq U'\mid |W|=C\,\},
\]
so
\begin{equation}\label{eq:A-part}
\bigoplus_{W\in\mathcal A}A_W
=
\bigoplus_{\substack{W\subseteq U'\\|W|=C}}A_W
=
S_C(U').
\end{equation}

\smallskip\noindent
\emph{(ii) The $\mathcal B$-part.}
Define a map
\[
\phi:\{\,W'\subseteq U'\mid |W'|=C-1\,\}\longrightarrow \mathcal B,
\qquad
\phi(W'):=W'\cup\{u\}.
\]
This map is a bijection:
\begin{itemize}
\item It is well-defined because if $W'\subseteq U'$ and $|W'|=C-1$, then
$u\notin W'$ and thus $|W'\cup\{u\}|=|W'|+1=C$, and $\phi(W')$ contains $u$.
\item It is injective because if $W'_1\cup\{u\}=W'_2\cup\{u\}$, removing $u$ gives
$W'_1=W'_2$.
\item It is surjective because if $W\in\mathcal B$, then $u\in W$ and setting
$W':=W\setminus\{u\}$ gives $W'\subseteq U'$ and $|W'|=C-1$, with $W=\phi(W')$.
\end{itemize}
Hence we may re-index the sum over $\mathcal B$ via $\phi$:
\begin{equation}\label{eq:B-reindex}
\bigoplus_{W\in\mathcal B}A_W
=
\bigoplus_{\substack{W'\subseteq U'\\|W'|=C-1}} A_{\phi(W')}
=
\bigoplus_{\substack{W'\subseteq U'\\|W'|=C-1}} A_{W'\cup\{u\}}.
\end{equation}
For each $W'\subseteq U'$ we have, by commutativity and associativity of $\otimes$,
\[
A_{W'\cup\{u\}}
=
\Bigl(\bigotimes_{x\in W'}\alpha_x \Bigr)\otimes \alpha_u
=
A_{W'}\otimes \beta.
\]
Substituting into \eqref{eq:B-reindex} and using distributivity of $\otimes$ over $\oplus$,
\begin{equation}\label{eq:B-part}
\bigoplus_{W\in\mathcal B}A_W
=
\bigoplus_{\substack{W'\subseteq U'\\|W'|=C-1}} (A_{W'}\otimes\beta)
=
\Bigl(\bigoplus_{\substack{W'\subseteq U'\\|W'|=C-1}} A_{W'}\Bigr)\otimes\beta
=
S_{C-1}(U')\otimes\beta.
\end{equation}

Finally, combining \eqref{eq:split-SC}, \eqref{eq:A-part}, and \eqref{eq:B-part}, we obtain
\[
S_C(U)=S_C(U')\ \oplus\ \bigl(S_{C-1}(U')\otimes\beta\bigr),
\]
as required.
\end{proof}
Finally we proceed with our proof for part $(i)$ of Theorem~\ref{th:correctness}.
\paragraph{Proof of (i)}
For any $\mathbb{K}$-instance $\hat I$, we prove the equality  $\langle Q^{\op C}_1\rangle^{\hat I}=\langle Q^{\op C}_2\rangle^{\hat I}$ by showing
  that for any arbitrary tuple in the query result the corresponding provenance expressions instantiated on any commutative $m$-semiring $\mathbb{K}$ under the assumptions of $(i)$, are same.
    $\langle Q^{\op C}_1\rangle^{\hat I}\text{ and }\langle Q^{\op C}_2\rangle^{\hat I}$ apply the same relational algebra operators to the same subquery $q$
    and the provenance semantics defined for those operators are identical. The two queries differ only in their implementation of the aggregate comparison (HAVING) condition; all other operators are interpreted identically. Hence, it
    suffices to show that for each group produced by
    the grouping operator, the provenance annotation produced because of the aggregate comparison coincides with
    that produced by the self-join query.
Since \(q\) is a binary query, every tuple in its output has the form
\((\#1,\#2)\).
Fix a value \(a\) of attribute \(\#1\), arbitrarily.

\smallskip\noindent\emph{Note: }\textit{For simplicity, we assume set semantics, i.e, $q$ produces no duplicate output tuples. In this setting, for a fixed group key $a$ selecting $C$ tuples whose second attributes form a strict join chain is the same as requiring the existence of $C-$ distinct tuples in the group. The same proof idea can be extended to multiset-semantics by treating duplicate tuples as distinct tuple occurences (e.g. by indexing inside groups) and refining the strict-ordering predicate in the join chain with this new tie-breaking attribute so that it is possible to select distinct tuple occurences with equal second attributes in increasing order.}

We define
\[
  U \;=\; \{\, (a,\#2) \mid (a,\#2) \in q \,\}.
\]
  That is, \(U\) is a set of tuples in \(q\) corresponding to group-key \(a\).
  Each \(u \in U\) has provenance
  \(\alpha_u \in \mathbb{K} \) and
  contributes the value \(1 \in M(\mathbb{N},+,.)\) to the count aggregate.

\smallskip\noindent\emph{Note: }\textit{Again, for multiset-semantics $U$ is to be defined as a finite index set of tuple-occurences of tuples in $q$ for a particular group-key. And each $u\in U $ will have provenance $\alpha_u\in\mathbb{K}$}

The provenance of the output tuple corresponding
  to group-key \(a\) depends exclusively on the tuples in $U$.
Tuples whose first attribute is different from \(a\) do not contribute to
this group and are therefore irrelevant for the computation of its
provenance.
Consider the query
\[
Q^{\op C}_1
=
\Pi_{\#1}\bigl(
\sigma_{\#2 \op C}(
  \gamma_{\#1}[1:+](q)
)\bigr).
\]
For the group with key value $a$, the operator \(\gamma_{\#1}[1:+]\) produces a single aggregate
\[
\mathdutchcal{a}
=
  \bigoplus_{\substack{\mathcal{M} \\ u \in U}} (\alpha_u,1)
\in \mathcal M .
\]
Applying the selection \(\sigma_{\#2 \op C}\) corresponds to evaluating the
aggregate-comparison predicate \([\mathdutchcal{a} \op C]\).
By our possible-world semantics, the provenance annotation of the output
tuple corresponding to group-key \(a\) is,
\begin{equation}\label{eq:Q1-prov}
\langle Q^{\op C}_1 \rangle^{\hat I}(a)
=
  \bigoplus_{W \subseteq U}
\Bigl(
  \bigotimes_{u\in W} \alpha_u
\;\otimes\;
  [|W|\op C]
\;\otimes\;
\bigl(
  \mathbb{1}_\mathbb{K} \ominus \bigoplus_{u \in U \setminus W} \alpha_u \bigr)
\Bigr).
\end{equation}
\\
For each subset \(W \subseteq U\) such that $|W|\op C$. Let,
\[
A_W:=\bigotimes_{w\in W}\alpha_w ,
\qquad
B_W:=\bigoplus_{w\in U\setminus W}\alpha_w .
\]
We define,
\[
T_U(W)\ :=\ A_W\ \otimes\ \bigl(\mathbb{1}_{\mathbb{K}} \ominus B_W\bigr)\ \in \mathbb K.
\]
By distributivity of $\otimes$ over $\ominus$ and then of $\otimes$ over $\oplus$,
\begin{align*}
T_U(W)
  &= (A_W\otimes \mathbb{1}_{\mathbb{K}})\ominus (A_W\otimes B_W) \\
  &= A_W\ominus\Bigl(A_W\otimes \bigoplus_{w\in U\setminus W}\alpha_w \Bigr) \\
&= A_W\ominus \bigoplus_{w\in U\setminus W}(A_W\otimes \alpha_w ) \\
&= A_W\ominus \bigoplus_{w\in U\setminus W}A_{W\cup\{w\}}.
\end{align*}
Since $[|W|\op C]=\mathbb{1}_\mathbb{K}$ iff for a fiven choice of $\op$, $|W|\op C$ is true and $\mathbb{0}_\mathbb{K}$ otherwise, \eqref{eq:Q1-prov} becomes
\begin{equation}\label{eq:pw_tw}
\langle Q^{\ge C}_1\rangle^{\hat I}(a)
=
\bigoplus_{\substack{W\subseteq U\\|W|\op C}} T_U(W).
\end{equation}
\smallskip\noindent\emph{Provenance of the join queries:}
For every $W\subseteq U $ and $C\ge 0$ we define,
\[
S_C \;:=\; \bigoplus_{\substack{W\subseteq U \\|W|=C}} A_W
\quad
(\text{with }S_0=\mathbb{1}_{\mathbb{K}}\text{ and }S_c=\mathbb{0}_\mathbb{K}\text{ if }C>|U|).
\]
Consider the query $Q^{\ge C}_2$.  A valuation contributing to the output
tuple with key $a$ corresponds to choosing $C$ tuples from $q$ with first
attribute $a$ and strictly increasing second attributes (due to the chain of
join conditions $\#2<\#4<\cdots$). This corresponds to choosing a
$C$-element subset $W\subseteq U$, and
$\Pi_{\#1}$ then takes the $\oplus$-sum over all these subsets.
Therefore
\[
\langle Q^{\ge C}_2\rangle^{\hat I}(a)
=
\bigoplus_{\substack{W\subseteq U \\|W|=C}} A_W
=
S_C.
\tag{$J_{\ge}$}
\]
By definition,
\[
Q^{=C}_2 = Q^{\ge C}_2 \ominus Q^{\ge (C+1)}_2,
\qquad
Q^{\le C}_2 = \Pi_{\#1}(q) \ominus Q^{\ge (C+1)}_2,
\]
Hence, using $(J_{\ge})$,
\[
\langle Q^{=C}_2\rangle^{\hat I}(a)
=
S_C \ominus S_{C+1},
\qquad
\langle Q^{\le C}_2\rangle^{\hat I}(a)
=
S_1 \ominus S_{C+1}.
\tag{$J_{=,\le}$}
\]

\smallskip\noindent\textbf{\emph{When $\op$ is $\ge$ :}}
From \eqref{eq:pw_tw} with $\op = \ge$ we set,
\[
F_C(U)\ :=\ \bigoplus_{\substack{W\subseteq U\\|W|\ge C}} T_U(W).
\]
Now,
\begin{equation}\label{eq:PW_ge_equals_F}
\langle Q^{\ge C}_1\rangle^{\hat I}(a)=F_C(U).
\tag{$PW_{\ge}$}
\end{equation}
We already showed for the join query that
\[
\langle Q^{\ge C}_2\rangle^{\hat I}(a)=S_C(U)
\quad\text{where}\quad
S_C(U)=\bigoplus_{\substack{W\subseteq U\\|W|=C}}A_W.
\tag{$J_{\ge}$}
\]
By Lemma~\ref{lem:FC_recurrence}, for any $u\in U$ with $U'=U\setminus\{u\}$ and $\beta=\alpha_u$,
and for all $C\ge 1$,
\[
F_C(U)=F_C(U')\ \oplus\ \bigl(F_{C-1}(U')\otimes\beta\bigr).
\tag{R$_F$}
\]
By Lemma~\ref{lem:SC_recurrence}, $S_C(U)$ satisfy the same recurrence:
\[
S_C(U)=S_C(U')\ \oplus\ \bigl(S_{C-1}(U')\otimes\beta\bigr)
\qquad(C\ge 1).
\tag{R$_S$}
\]
We prove that $F_C(U)=S_C(U)$ by induction on $|U|$,  for all finite $U$ and all $C\ge 1$.
If $|U|=0$ then $U=\varnothing$ and $F_C(\varnothing)=\mathbb 0=S_C(\varnothing)$ for $C\ge 1$
(empty $\oplus$-sum).
Assume $|U|>0$ and pick $u\in U$, let $U'=U\setminus\{u\}$.
By the induction hypothesis, $F_C(U')=S_C(U')$ and $F_{C-1}(U')=S_{C-1}(U')$.
Applying (R$_F$) and (R$_S$) gives
\[
F_C(U)
=F_C(U')\oplus(F_{C-1}(U')\otimes\beta)
=S_C(U')\oplus(S_{C-1}(U')\otimes\beta)
=S_C(U).
\]
Thus $F_C(U)=S_C(U)$ for all $C\ge 1$, hence by \eqref{eq:PW_ge_equals_F} and $(J_{\ge})$,
\[
\langle Q^{\ge C}_1\rangle^{\hat I}(a)=\langle Q^{\ge C}_2\rangle^{\hat I}(a).
\]
\smallskip\noindent\textbf{\emph{When $\op$ is $=$ :}}
For each $n\in\mathbb N$ we have the boolean identity
\[
\chi_{=}(n,C)=\chi_{\ge}(n,C)\ \ominus\ \chi_{\ge}(n,C+1),
\]
since $\chi_{\ge}(n,C),\chi_{\ge}(n,C+1)\in\{\mathbb 0,\mathbb 1\}$.
Let
\[
P_W \ :=\ A_W\otimes(\mathbb 1\ominus B_W)=T_U(W).
\]
Then for each $W\subseteq U$,
\[
P_W\otimes \chi_{=}( |W|,C)
=
P_W\otimes\bigl(\chi_{\ge}(|W|,C)\ominus\chi_{\ge}(|W|,C+1)\bigr)
=
(P_W\otimes\chi_{\ge}(|W|,C))\ominus(P_W\otimes\chi_{\ge}(|W|,C+1)),
\]
by distributivity of $\otimes$ over $\ominus$.
Summing over $W$ and using right-distributivity of $\ominus$ over $\oplus$
(which holds because $\mathbb K$ is idempotent, ),
we obtain
\begin{align*}
\langle Q^{=C}_1\rangle^{\hat I}(a)
&=
\bigoplus_{W\subseteq U}\bigl(P_W\otimes\chi_{=}( |W|,C)\bigr)\\
&=
\left(\bigoplus_{W\subseteq U} P_W\otimes\chi_{\ge}(|W|,C)\right)
\ominus
\left(\bigoplus_{W\subseteq U} P_W\otimes\chi_{\ge}(|W|,C+1)\right)\\
&=
\langle Q^{\ge C}_1\rangle^{\hat I}(a)\ \ominus\ \langle Q^{\ge (C+1)}_1\rangle^{\hat I}(a).
\end{align*}
By the already established $\ge$-case,
$\langle Q^{\ge C}_1\rangle^{\hat I}(a)=S_C$ and $\langle Q^{\ge (C+1)}_1\rangle^{\hat I}(a)=S_{C+1}$,
hence
\[
\langle Q^{=C}_1\rangle^{\hat I}(a)=S_C\ominus S_{C+1}=\langle Q^{=C}_2\rangle^{\hat I}(a)
\]
by $(J_{=,\le})$.

\smallskip\noindent\textbf{\emph{When $\op$ is $\le$:}}
For $n\ge 1$ we have the boolean identity
\[
\chi_{\le}(n,C)=\chi_{\ge}(n,1)\ \ominus\ \chi_{\ge}(n,C+1).
\]
(Indeed, for $n\ge 1$ the condition $n\le C$ is equivalent to $n\ge 1$ and not $n\ge C+1$.)
Since groups produced by $\gamma_{\#1}$ are nonempty, only $n=|W|\ge 1$ occur.
Repeating the same lifting argument as above gives
\[
\langle Q^{\le C}_1\rangle^{\hat I}(a)
=
\langle Q^{\ge 1}_1\rangle^{\hat I}(a)\ \ominus\ \langle Q^{\ge (C+1)}_1\rangle^{\hat I}(a).
\]
By the $\ge$-case, $\langle Q^{\ge 1}_1\rangle^{\hat I}(a)=S_1=\bigoplus_{x\in U}\alpha_x $ and
$\langle Q^{\ge (C+1)}_1\rangle^{\hat I}(a)=S_{C+1}$, hence
\[
\langle Q^{\le C}_1\rangle^{\hat I}(a)
=
\Bigl(\bigoplus_{x\in U}\alpha_x \Bigr)\ominus S_{C+1}
=
\langle Q^{\le C}_2\rangle^{\hat I}(a)
\]
by $(J_{=,\le})$.

\smallskip
Since the equalities hold for each group key $a$, we conclude
$\langle Q^{\op C}_1\rangle^{\hat I}=\langle Q^{\op C}_2\rangle^{\hat I}$ for
$\op\in\{\ge,=,\le\}$.
\paragraph{Conclusion.}
Comparing the expressions obtained for
\(\langle Q^{\op C}_1 \rangle^{\hat I}(a)\) and
\(\langle Q^{\op C}_2 \rangle^{\hat I}(a)\),
we conclude that they are equal for the fixed value \(a\).
Since \(a\) was arbitrary, the two queries produce identical provenance
annotations for all output tuples.
Therefore,
\[
\langle Q^{\op C}_1 \rangle^{\hat I}
=
\langle Q^{\op C}_2 \rangle^{\hat I}.
\qed
\]\paragraph{Proof of (ii)}
\paragraph{Proof of (iii)}
\section{Implementation}
Using our semantics, we obtain the expression $[u\oslash v]$. Without loss of generality, 
for every constant value of the aggregation expression $v=C$, we can rewrite this expression 
as $[u\oslash C]_{\mathrm{op}}$. We now present a dynamic programming algorithm~\ref{alg:dp_subsets} to compute the set of subsets 
that satisfy the condition imposed by the operator $\mathrm{op}\in\{=,>,<,\ge,\le\}$ with 
respect to the constant $C$. Using this set of subsets, we can compute the probability of 
$[u\oslash C]_{\mathrm{op}}$ 
by using the Algorithm~\ref{alg:probability}.
\begin{algorithm}
  \label{alg:dp_subsets}
  \caption{DP algorithm for $[\mathdutchcal{u}\oslash C]_{\mathrm{op}}$ (covers $=,>,<,\ge,\le$)}
  \begin{algorithmic}[1]
    \REQUIRE a dictionary $A \leftarrow \mathrm{supp}(\mathdutchcal{u})$, a constant $C \in M$, and an operator $\mathrm{op}\in\{=,>,<,\ge,\le\}$
    \ENSURE a list $\mathcal{R}$ of subsets of $A$, s.t.\ $\forall R \in \mathcal{R}$, $\ \sum_{\mathbf{x}\in R}\mathbf{x}^{(M)}\ \mathrm{op}\ C$

    \IF{$\mathrm{op}\in\{>,\ge\}$}
      \STATE $T \gets \sum_{\mathbf{x}\in A}\mathbf{x}^{(M)}$ \COMMENT{max possible sum}
      \STATE $J \gets T$
    \ELSIF{$\mathrm{op}=<$}
      \STATE $J \gets C-1$
    \ELSE
      \STATE $J \gets C$ \COMMENT{$\mathrm{op}\in\{=,\le\}$}
    \ENDIF

    \STATE Initialize $dp[0] \leftarrow \{\emptyset\}$
    \FOR{$j \gets 1$ \TO $J$}
      \STATE $dp[j] \leftarrow \emptyset$
    \ENDFOR

    \FOR{$i \gets 1$ \TO $|A|$}
      \STATE $w \gets A^{(M)}[i]$, $g \gets A^{(K)}[i]$
      \FOR{$j \gets J, J-1, \dots, w$}
        \STATE $p \gets j - w$
        \IF{$dp[p] \neq \emptyset$}
          \FORALL{$S \in dp[p]$}
            \STATE append $S \cup \{g\}$ to $dp[j]$
          \ENDFOR
        \ENDIF
      \ENDFOR
    \ENDFOR

    \IF{$\mathrm{op} = =$}
      \STATE $\mathcal{R} \leftarrow dp[C]$
    \ELSIF{$\mathrm{op} = >$}
      \STATE $\mathcal{R} \leftarrow \bigcup_{j=C+1}^{J} dp[j]$
    \ELSIF{$\mathrm{op} = <$}
      \STATE $\mathcal{R} \leftarrow \bigcup_{j=1}^{C-1} dp[j]$
    \ELSIF{$\mathrm{op} = \ge$}
      \STATE $\mathcal{R} \leftarrow \bigcup_{j=C}^{J} dp[j]$
    \ELSE % \mathrm{op}=\le
      \STATE $\mathcal{R} \leftarrow \bigcup_{j=1}^{C} dp[j]$
    \ENDIF

    \STATE \RETURN $\mathcal{R}$
  \end{algorithmic}
\end{algorithm}


\begin{algorithm}
  \label{alg:probability}
  \caption{Probability computation for $[\mathdutchcal{u}\oslash C]_{\mathrm{op}}$}
  \begin{algorithmic}[1]
    \REQUIRE dictionary $A\leftarrow\mathrm{supp}(\mathdutchcal{u})$, constant $C\in M$, operator $\mathrm{op}$, and probabilities $p(g)$ for all $g\in\mathcal{G}$
    \ENSURE $\Pr\big([\mathdutchcal{u}\oslash C]_{\mathrm{op}}\big)$
    \STATE $\mathcal{R} \gets \textsc{DP-Subsets}(A,C,\mathrm{op})$ \COMMENT{Algorithm above}
    \STATE $P \gets 0$
    \FORALL{$S \in \mathcal{R}$}
      \STATE $\pi \gets 1$
      \FORALL{$g \in \mathcal{G}$}
        \IF{$g \in S$}
          \STATE $\pi \gets \pi \cdot p(g)$
        \ELSE
          \STATE $\pi \gets \pi \cdot (1 - p(g))$
        \ENDIF
      \ENDFOR
      \STATE $P \gets P + \pi$
    \ENDFOR
    \STATE \RETURN $P$
  \end{algorithmic}
\end{algorithm}

% \begin{table}[hbt]
%   \centering
%   \begin{tabular}{cc}
%     \toprule
%     operator & $\mathcal{R}$\\
%     \midrule
%     $=$ & $dp[C]$\\
%     $>$ & $\bigcup_{j=C+1}^{T} dp[j]$\\
%     $<$ & $\bigcup_{j=1}^{C-1} dp[j]$\\
%     $\geq$ & $\bigcup_{j=C}^{T} dp[j]$\\
%     $\leq$ & $\bigcup_{j=1}^{C} dp[j]$\\
%     \bottomrule
%   \end{tabular}
% \end{table}

\begin{algorithm}
  \label{alg:count-prov-comparisons}
  \caption{Provenance for \texttt{COUNT(*)} comparisons}
  \begin{algorithmic}[1]
    \REQUIRE Support set $A=\{a_1,\dots,a_n\}$ of tuple annotations; threshold $m$
    \ENSURE $\mathrm{Prov}(\texttt{COUNT(*)} \ge m)$, $\mathrm{Prov}(\texttt{COUNT(*)} \le m)$, $\mathrm{Prov}(\texttt{COUNT(*)} = m)$

    \STATE \textbf{Function} $\textsc{ProdSet}(S)$
    \STATE \hspace{1em} $p \gets \mathbf{1}$
    \FORALL{$x \in S$}
      \STATE \hspace{1em} $p \gets p \otimes x$
    \ENDFOR
    \STATE \hspace{1em} \textbf{return} $p$

    \STATE $P_{\ge m} \gets \mathbf{0}$
    \FORALL{$S \subseteq A$ such that $|S| = m$}
      \STATE $P_{\ge m} \gets P_{\ge m} \oplus \textsc{ProdSet}(S)$
    \ENDFOR

    \STATE $P_{\le m} \gets \mathbf{0}$
    \FORALL{$S \subseteq A$ such that $|S| = n-m$}
      \STATE $S' \gets \{\bar{x}\mid x\in S\}$
      \STATE $P_{\le m} \gets P_{\le m} \oplus \textsc{ProdSet}(S')$
    \ENDFOR

    \STATE $P_{=m} \gets P_{\ge m} \otimes P_{\le m}$

    \STATE \RETURN $(P_{\ge m}, P_{\le m}, P_{=m})$
  \end{algorithmic}
\end{algorithm}

\section{Experiments}

\section{Conclusion}

\end{document}
