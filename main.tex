\documentclass{article}
\usepackage{graphicx} % Required for inserting images
\usepackage{amssymb}
\usepackage{amsmath}
\usepackage{mathrsfs}
\usepackage[bb=boondox]{mathalfa}
\makeatletter
\newcommand{\bigplus}{%
  \DOTSB\mathop{\mathpalette\mattos@bigplus\relax}\slimits@
}
\newcommand\mattos@bigplus[2]{%
  \vcenter{\hbox{%
    \sbox\z@{$#1\sum$}%
    \resizebox{!}{0.9\dimexpr\ht\z@+\dp\z@}{\raisebox{\depth}{$\m@th#1+$}}%
  }}%
  \vphantom{\sum}%
}
\makeatother
\title{Provenance of aggregate queries with HAVING clause in ProvSQL.}
\author{Pratik Karmakar \and Aryak Sen \and Pierre Senellart}
\date{}

\begin{document}

\maketitle

$R$ is a relation on set of attributes $U$.\\
We consider $U^{GB} \subseteq U$ and $U^{AGG}\subseteq U$ and $U^{GB}\cap U^{AGG} = \phi$.\\
For each tuple $t$, 
\begin{align*}
T = \{t^*\in \mathrm{Supp}(R)|\forall u\in U^{GB}, t(u) = t^*(u)\}.    
\end{align*}

Extending on the semantics of aggregate GROUP BY queries~\cite{amsterdamer2011provenance} we express the provenance of HAVING queries as: 
\begin{align*}
    q:= \sigma_{SUM} = c
\end{align*}

\begin{align*}
\mathrm{Provenance}(q) = \delta(\bigplus_{t_i \in T} t_i) * [\bigplus{K\otimes{SUM}}_{t_i\in T}t_i \otimes c_i = c\otimes \mathbb{1}]    
\end{align*}

\footnote{In deterministic scenario, provenance of HAVING queries is just Boolean existence onus the comparison operator??}
\footnote{For probabilistic databases, provenance of HAVING queries is to be computed using the DP algo.}
\bibliographystyle{splncs04}
\bibliography{ref}
\end{document}
