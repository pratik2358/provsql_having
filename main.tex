\documentclass{article}
\usepackage{graphicx} % Required for inserting images
\usepackage{amssymb}
\usepackage{amsmath}
\usepackage{mathrsfs}
\usepackage{csquotes}
\usepackage[bb=boondox]{mathalfa}
\newcommand{\dq}[1]{\text{\enquote{#1}}}
\makeatletter
\newcommand{\bigplus}{%
  \DOTSB\mathop{\mathpalette\mattos@bigplus\relax}\slimits@
}
\newcommand\mattos@bigplus[2]{%
  \vcenter{\hbox{%
    \sbox\z@{$#1\sum$}%
    \resizebox{!}{0.9\dimexpr\ht\z@+\dp\z@}{\raisebox{\depth}{$\m@th#1+$}}%
  }}%
  \vphantom{\sum}%
}
\makeatother
\title{Provenance of aggregate queries with HAVING clause in ProvSQL.}
\author{Pratik Karmakar \and Aryak Sen \and Pierre Senellart}
\date{}

\begin{document}

\maketitle

$R$ is a relation on set of attributes $U$.\\
We consider $U^{GB} \subseteq U$ and $U^{AGG}\subseteq U$ and $U^{GB}\cap U^{AGG} = \phi$.\\
For each tuple $t$, 
\begin{align*}
T = \{t^*\in \mathrm{Supp}(R)|\forall u\in U^{GB}, t(u) = t^*(u)\}.    
\end{align*}

Extending on the semantics of aggregate GROUP BY queries~\cite{amsterdamer2011provenance} we express the provenance of HAVING queries as: 
\begin{align*}
    q:= \sigma_{SUM} = c
\end{align*}

\begin{align*}
\mathrm{Provenance}(q) = \delta(\bigplus_{t_i \in T} t_i) * [\bigplus{K\otimes{SUM}}_{t_i\in T}t_i \otimes c_i = c\otimes \mathbb{1}]    
\end{align*}

\section{Formula Semiring}
\begin{align*}
  \mathcal{K}_{formula} = (K, \oplus, \otimes, \mathbb{0}, \mathbb{1}, \delta, \ominus)
\end{align*}
$K \leftarrow$ Set of strings\\
\begin{align*}
\oplus (k_1,\dots, k_n) = \begin{cases}
    0_k & \text{if } n = 0\\
    k_1 & \text{if } n = 1\\
    \dq{(}+k_1 + \oplus + k_2 + \dots + \oplus + k_n +) & \text{if } n > 1  
\end{cases}
\end{align*}
\begin{align*}
  \otimes: K^n \rightarrow K \\
\end{align*}
\begin{align*}
  \ominus: K\times K \rightarrow K: k_1 \ominus k_2 = (+k_1+ \ominus +k_2+)
\end{align*} 
\begin{align*}
\delta: K \rightarrow K\\
\delta(k) = \begin{cases}
    \delta(+k+) & \text{if } k[0] \neq 'C'\\
    \delta +k & \text{if } k[0] = 'C'\\
    \end{cases}
\end{align*}
\begin{align*}
  &Cmp: K^2 \rightarrow K\\
  &op\in\{=,\neq, <, \leq, >, \geq\}\\
  &Cmp(k_1, op, k_2) = [+k_1 + op + k_2 +]
\end{align*}
\begin{align*}
&agg(\gamma, \{q_i\}): \{SUM, MIN, MAX, PROD\}\times K^n \rightarrow K\\
&SUM: \sum q_i\\
&MIN: min(k_1, \dots, k_n)\\
&MAX: max(k_1, \dots, k_n)\\
&PROD: \prod k_i
\end{align*}
$U$ be the set of attributes on domain $D$.\\
Tuples are $tup(U) = \{t: U \rightarrow D\}$
A K-relation is $R: tup(U) \rightarrow K$

\begin{itemize}
  \item Empty relation: $[Q](t) = 0_K$
  \item SELECTION: $[\sigma_{\theta}(R)](t) = \delta(Cmp(\theta_1)) \otimes \delta(Cmp(\theta_2))\otimes \dots \otimes R(t)$
  \item NATURAL JOIN: $[R \bowtie S](t) = R(t_R) \otimes S(t_S)$, given $Q = R \bowtie S \forall t$ with projection $t_R, t_S$
  \item RENAME: $\rho_{A\rightarrow B}(R), [\rho_{A\rightarrow B} (R)](t) = R(t[A \mapsto B])$
  \item DIFF: $[R - S](t) = R(t) \ominus S(t)$
  \item PROJECTION: $U \subseteq Schema(R), u\in tup(U)$\\
  $[\Pi_U(R)](u) = \oplus_{t[U] = u} R(t)$\\
  \item UNION: $[R \cup S](t) = R(t) \oplus S(t)$
\end{itemize}

Let $G$ be the set of attributes in a GROUP BY clause,
$$G \subseteq Schema(R)$$
and aggregate function $f\in\{SUM, PROD, MIN, MAX\}$ over attributes $A\in U\setminus G$.\\
And, $$\delta = \gamma_{G,f(A)}(R),$$
$$\delta(g) = agg(\gamma, \{R(t) | t[G] = g\}) \forall group \in tup(G)$$

\subsection{HAVING only for constant C to be compared against $f$}
\begin{align*}
  \delta = \gamma_{G,f(A)}(R): g \mapsto \delta(g) = f(\{R(t)|t[G] = g\})\in K
\end{align*}

Only one provenance token $\delta(g)$ for each group $g$ in the GROUP BY clause.\\
We give semantics for $f(A) op C$:
\begin{align*}
  [\gamma_{G,f(A)}(R) \text{ HAVING } f(A) op C] = (g \mapsto \gamma ([\delta(g) op C])\otimes \delta(g))\\
  \text{where } op \in \{=, \neq, <, \leq, >, \geq\}
\end{align*}

\footnote{In deterministic scenario, provenance of HAVING queries is just Boolean existence onus the comparison operator??}
\footnote{For probabilistic databases, provenance of HAVING queries is to be computed using the DP algo.}
\bibliographystyle{splncs04}
\bibliography{ref}
\end{document}
