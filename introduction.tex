\section{Introduction}
Nested aggregations are an important feature to
have in a practical system. Basic support of nested aggregation
is already implemented in ProvSQL. However, there is an
important semantics issue: the approach presented in the
literature for supporting nested aggregation, which adds a
comparison operator, makes sense for some specific provenance
frameworks (such as Boolean provenance), but there is no
clear semantics of how these comparison operators should
be implemented for arbitrary m-semirings.

We provide the following contributions in this paper:
\begin{itemize}
\item We introduce a new semantics for the provenance of HAVING queries
  (i.e., of queries involving selection on an aggregate value) in any
  arbitrary commutative m-semiring and for any aggregate
  function (not necessarily commutative or based on a monoid).

\item We show that this semantics coincides with a natural JOIN-based
  semantics for COUNT(*) HAVING queries, in the case where the m-semiring
  is idempotent and $\otimes$ distributes over $\ominus$. This is in
  particular the case in the Boolean semiring, which implies that our
  semantics is compatible with probabilistic query evaluation.

\item We provide a characterization of most commutative m-semirings
  studied in the literature, which for each a formal proof in the Lean
  assistant of the validity of their definition (including the $\ominus$
  operator) as well as whether they are idempotent and have
  distributivity of $\otimes$ over $\ominus$. This is important as this
  fixes wrong claims made in previous papers \cite{Geerts,limitations}.

\item We provide algorithms implementing our semantics: a general
  algorithm based on possible-world enumeration; algorithms specialized
  for COUNT, SUM, MIN, and MAX aggregation operators; and optimizations
  thereof to obtain a lower parameterized complexity for some comparison
  operators and some m-semirings (especially, absorptive ones).

\item We implement these algorithms within the ProvSQL
  provenance-tracking extension of PostgreSQL, and explain the relevant
  implementation issues.

\item We describe experiments which show
  the viability of the approach on a real-world database, including for probabilistic
  query evaluation.
\end{itemize}
