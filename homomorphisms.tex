\subsection{Homomorphisms}

\subsubsection{$\Phi : \mathcal{B} \to T$}

$$T = (\mathcal{T}, \cup, \cap, \emptyset, \{]-\infty, +\infty[\}, \backslash)$$

For $A,B\in \mathcal{T}$, we define $A\backslash B$ as:
$$A\backslash B = \cup_{a\in A, b\in B}a\cap \bar{b}$$ 
where $\bar{b} = ]-\infty, +\infty[\backslash b$.

Now for Boolean function semiring,
$$\mathcal{B} = (\textbf{b}, \lor, \land, 0_{\mathcal{B}}, 1_{\mathcal{B}})$$
Where $\textbf{b} = (b_1,\dots,b_n) \in \{0,1\}^n$ is the set of all Boolean functions on $n$ variables, $\lor$ 
is pointwise disjunction, $\land$ is pointwise conjunction, $0_{\mathcal{B}}$ is the constant function that is always false, and $1_{\mathcal{B}}$ is the constant function that is always true.

where $0_B$ is the constant function that is always false, and $1_B$ is the constant function that is always true.
For each Boolean valuation $\textbf{b}$, define:

$$
Q_b \subseteq \mathbb{R} \footnote{- \(Q_b\) intervals basically tell us when each combination of tuple truth values happens.
- \(\Phi(f)\) simply collects all the times when \(f\) would be true for each tuple.}
$$


$Q_b$ form a disjoint cover of time.

We define:

$$
\Phi(f) = \bigcup_{\substack{b \in \{0,1\}^n \\ f(b) = 1}} Q_b
$$

\begin{proof}
\begin{itemize}
\item \textbf{Zero}
$$
\Phi(0_{\mathcal{B}}) = \bigcup_{b:\{0,1\}^n \mapsto \{0\}} Q_b = \emptyset
$$

\item
\textbf{One}
$$
\Phi(1_{\mathcal{B}}) = \bigcup_{b:\{0,1\}^n \mapsto \{1\}} Q_b = ]-\infty, +\infty[
$$
\item
\textbf{Addition}
$$
\Phi(f \lor g)
= \bigcup_{b: f(b) \lor g(b) = 1} Q_b
= \left(\bigcup_{b: f(b) = 1} Q_b\right) \cup \left(\bigcup_{b: g(b) = 1} Q_b\right)
= \Phi(f) \cup \Phi(g)
$$
\item
\textbf{Multiplication}
$$
\Phi(f \land g)
= \bigcup_{b: f(b) \land g(b) = 1} Q_b
= \left(\bigcup_{b: f(b) = 1} Q_b\right) \cap \left(\bigcup_{b: g(b) = 1} Q_b\right)
= \Phi(f) \cap \Phi(g)
$$
\item
\textbf{monus}
$$
\Phi(f \land \neg g)
= \bigcup_{b: f(b) \land \neg g(b) = 1} Q_b
= \left(\bigcup_{b: f(b) = 1} Q_b\right) \cap \left(1_T \backslash \left(\bigcup_{b: g(b) = 1} Q_b\right)\right)
= \Phi(f) \backslash \Phi(g)
$$
    

\end{itemize}

\end{proof}

Thus $\Phi$ is a semiring homomorphism.



