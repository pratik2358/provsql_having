\section{FOO}
\subsection{Homomorphisms}
We have seen that our provenance semantics work for 
the semiring of Boolean functions. In this section we investigate other m-semirings and see if they are homomorphic 
to the semiring of Boolean functions. This will ensure that the provenance sematics, that worked for $B[X]$ can be transferred to 
our target semirings without re-deriving from scratch.


\subsubsection{$\Phi^{\mathrm{T}}_v : B[X] \to T$}
For any valuation $v:X \to T$,  $\exists$ a homomorphism $\Phi^{\mathrm{T}}_v : B[X] \to T$ such that for all $x\in X$, 
$\Phi^{\mathrm{T}}_v(x\mapsto [x]_{B[X]})=v(x)$.

\subsubsection*{Existence(by construction)}
\begin{proof}
Let $\mathsf{Form}(X)$ be the set of well formed Boolean formulas built on the variables in X. There is a canonical surjection 
$q:\mathsf{Form}(X)\rightarrow B[X]$(basically maps each syntactical formula to its equivalence class).

We define $\iota: X \rightarrow \mathsf{Form}(X)$, and thus essentially any Boolean function in $B[X]$ is a composition $q\circ \iota$.

Define a map $\Phi^*:\mathsf{Form}(X)\rightarrow T$ by recursion:
\begin{align*}
&\Phi^*(0_B) := \varnothing,\quad\Phi^*(1_B) := \mathbb{R},\\
&\Phi^*(\iota(x)) := v(x)\quad\text{for }x\in X,\\
\vspace{2cm}
&\text{And for any }\varphi,\psi \in \mathsf{Form}(X):\\
&\Phi^*(\varphi\vee\psi) := \Phi^*(\varphi)\cup\Phi^*(\psi),\\
&\Phi^*(\varphi\wedge\psi) := \Phi^*(\varphi)\cap\Phi^*(\psi),\\
&\Phi^*(\neg\varphi) := \mathbb{R}\setminus\Phi^*(\varphi)= \Phi^*(\varphi)^\complement\\
&\Phi^*(\phi \setminus_B \psi) = \Phi^*(\phi \wedge \neg \psi) = \Phi^*(\phi)\cap \Phi^*(\neg \psi)
\end{align*}
such that the Boolean connectives are interpreted in $T$ through the set-theoretic operations $\cup, \cap$ and set complement.

$\mathsf{Form}(X)$ being inductively generated, $\Phi^*$ is well-defined on all formulae.
  For any two logically equivalent formulae $\phi, \psi \in \mathsf{Form}(X)$ we have, $\Phi^*(\phi)=\Phi^*(\psi)$ in $T$. This is because every Boolean algebra axiom in $Form(X)$  is interpreted in $T$ through set-theoretic operations and since
  $\mathsf{ker}(\Phi^*):=\{(\phi,\psi)\in \mathsf{Form}(X)\times \mathsf{Form}(X)|\Phi^*(\phi)=\Phi^*(\psi)\}$ is a Boolean congrunence satisfying the equational theory of Boolean algebra, 
  therefore logically equivalent formulae are mapped to the same element in $T$ by $\Phi^*$.
  Thus for any Boolean formula $\phi$, we can define a homomorphism on it's equivalence class(which is essentially a Boolean function) as,  $\Phi_v^T(q(\phi)):=\Phi^*(\phi)$. This ensures that $\Phi_v^T$ is well-defined and for 
  any variable $x \in X$, $\Phi_v^T(x\mapsto [x]_{B[X]})=\Phi_v^T(q(\iota(x)))=\Phi^*(\iota(x))=v(x)$.

\end{proof}

\subsubsection*{Uniqueness}
\begin{proof}
Assume that $\exists\;\Psi_v^T:B[x]\rightarrow T$ such that $\Psi_v^T(x\mapsto [x]_{B[X]})=v(x)\text{ for all }x \in X$.

$B[X]$ is a free Boolean algebra on $X$  and $\alpha \in B[X]$ is image of $q:\mathsf{Form}(X)\rightarrow B[X]$ of some boolean formula $\phi$. 
When it comes to free algebras, homomorphisms are only determined by the values of their generators(all Boolean atomic functions, $[x]_{B[x]}\in B[X]$)
And by assumption of $\Psi_v^T$ and existence of $\Phi_v^T$, they agreee on all generators. Hence, $\Psi_v^T=\Phi_v^T$.
\end{proof}

\subsubsection*{$\Phi_v^T$ preserves Axiom A.13.}

Let $\phi, \psi, \gamma \in B[X]$ and pick representatives $\tilde{\phi}, \tilde{\psi}, \tilde{\gamma} \in \mathsf{Form}(X)$ such that
\[
q(\tilde{\phi}) = \phi, \quad q(\tilde{\psi}) = \psi, \quad q(\tilde{\gamma}) = \gamma.
\]

Then we have
\[
\begin{aligned}
\Phi_v^T\big(\phi \wedge (\psi \setminus_B \gamma)\big) 
&= \Phi^*\big(\tilde{\phi} \wedge (\tilde{\psi} \wedge \neg \tilde{\gamma})\big) \\
&= \Phi^*(\tilde{\phi}) \cap \Phi^*(\tilde{\psi} \wedge \neg \tilde{\gamma}) \\
&= \Phi^*(\tilde{\phi}) \cap \big(\Phi^*(\tilde{\psi}) \cap \Phi^*(\neg \tilde{\gamma})\big) \\
&= \Phi^*(\tilde{\phi}) \cap \Phi^*(\tilde{\psi}) \cap (\Phi^*(\tilde{\gamma}))^\complement.
\end{aligned}
\]

Similarly,
\[
\begin{aligned}
\Phi_v^T\big((\phi \wedge \psi) \setminus_B (\phi \wedge \gamma)\big) 
&= \Phi^*\big((\tilde{\phi} \wedge \tilde{\psi}) \wedge \neg (\tilde{\phi} \wedge \tilde{\gamma})\big) \\
&= (\Phi^*(\tilde{\phi}) \cap \Phi^*(\tilde{\psi})) \cap (\Phi^*(\tilde{\phi} \wedge \tilde{\gamma}))^\complement \\
&= (\Phi^*(\tilde{\phi}) \cap \Phi^*(\tilde{\psi})) \cap (\Phi^*(\tilde{\phi}) \cap \Phi^*(\tilde{\gamma}))^\complement \\
&= (\Phi^*(\tilde{\phi}) \cap \Phi^*(\tilde{\psi})) \cap (\Phi^*(\tilde{\phi})^\complement \cup \Phi^*(\tilde{\gamma})^\complement) \\
&= (\underbrace{\Phi^*(\tilde{\phi}) \cap \Phi^*(\tilde{\psi}) \cap \Phi^*(\tilde{\phi})^\complement}_{\varnothing}) 
   \;\cup\; (\Phi^*(\tilde{\phi}) \cap \Phi^*(\tilde{\psi}) \cap \Phi^*(\tilde{\gamma})^\complement) \\
&= \Phi^*(\tilde{\phi}) \cap \Phi^*(\tilde{\psi}) \cap (\Phi^*(\tilde{\gamma}))^\complement.
\end{aligned}
\]

Hence,
\[
\Phi_v^T\big(\phi \wedge (\psi \setminus_B \gamma)\big) = \Phi_v^T\big((\phi \wedge \psi) \setminus_B (\phi \wedge \gamma)\big),
\]
which shows that $\Phi_v^T$ preserves Axiom A13.


\subsubsection{$\Phi_{\mathrm{trop}}: B^X \to T_{\min}$}

We define:
$$
T_{\min} = (\mathbb{R}_{\ge 0} \cup \{\infty\},\min,+,\infty,0,\setminus_T).
$$
\begin{itemize}
    \item Addition: \(\min\)
    \item Multiplication: \(+\)
    \item Zero: \(\infty\)
    \item One: \(0\)
    \item Monus:
    \[
    a \setminus_T b :=
    \begin{cases}
        a & \text{if } a < b \\
        \infty & \text{otherwise}
    \end{cases}
    \]
\end{itemize}



\textbf{Homomorphism}:\newline
Assign each token \(x_i\in X\) a cost \(c(x_i)\in\mathbb{R}_{\ge0}\cup\{\infty\}\). For any valuation \(\nu\), define
$$
w(\nu) = \sum_{x_i:\nu(x_i)=\top} c(x_i).
$$
Then for boolean function  \(\varphi\):
$$
\Phi_{\mathrm{trop}}(\varphi) 
= \min_{\nu:\varphi(\nu)=\top} w(\nu),
$$
with \(\Phi_{\mathrm{trop}}(0_B)=\infty\).
\newline

\textbf{Proof sketch}: \newline
- \(\Phi_{\mathrm{trop}}(\varphi \vee \psi) = \min(\Phi_{\mathrm{trop}}(\varphi), \Phi_{\mathrm{trop}}(\psi))\).  
- \(\Phi_{\mathrm{trop}}(\varphi \wedge \psi) = \Phi_{\mathrm{trop}}(\varphi) + \Phi_{\mathrm{trop}}(\psi)\).  
- Monus: follows. \newline
\textbf{Example}: \newline
Let \(X=\{x_1,x_2\}\), \(c(x_1)=2\), \(c(x_2)=5\).  
- \(\varphi = x_1\): \(\Phi_{\mathrm{trop}}(\varphi)=2\).  
- \(\psi = x_2\): \(\Phi_{\mathrm{trop}}(\psi)=5\).  
- \(\varphi \setminus_B \psi\) true only when \(\nu=(\top,\bot)\), cost \(2\).

It is easy to show that $A13$ and absorptivity are satisfied.


\subsubsection{$\Phi_{\mathrm{sec}}: B^X \to T_{\sec}$}

For each \(x\in X\) , say, there is a clearance level \(\lambda(x)\in \mathcal L\) assigned to it,
with \(\mathcal L\) totally ordered: \(\ell_{\min}\) (least restrictive),
and \(\ell_{\max}\) (most restrictive).

For a valuation \(\nu\):
\[
S(\nu) = \{x\in X \mid \nu(x)=\top\}.
\]
\[
\sec(\nu) =
\begin{cases}
\ell_{\min}, & S(\nu)=\emptyset,\\
\max_{x\in S(\nu)} \lambda(x), & \text{o.w.}
\end{cases}
\]



Then we can define a homomorphism:

\[
\Phi_{\mathrm{sec}}(\varphi)
\;=\;
\min_{\nu:\varphi(\nu)=\top}\;\sec(\nu),
\]
and if \(\varphi\equiv 0\), then
\[
\Phi_{\mathrm{sec}}(0_B) = \ell_{\max}.
\]

This is basically the “minimum clearance level required to get a tuple as a result.”



\textbf{Proof sketch:}
\begin{itemize}
    \item \textbf{Zero and One}  
        \begin{itemize}
            \item \(\Phi_{\mathrm{sec}}(0_B)=\ell_{\max}\) (no world satisfies 0).
            \item \(\Phi_{\mathrm{sec}}(1_B)=\ell_{\min}\) (world with no tokens gives \(\ell_{\min}\)).
        \end{itemize}
    \item \textbf{Addition} (\(\vee \to \min\))  
        \[
            \Phi(\varphi\vee\psi)
            = \min\bigl(\Phi(\varphi),\Phi(\psi)\bigr).
        \]
    \item \textbf{Multiplication} (\(\wedge \to \max\))  
        
        \[
            \Phi(\varphi\wedge\psi)
            = \max\bigl(\Phi(\varphi),\Phi(\psi)\bigr).
        \]
    \item \textbf{Monus} (\(\setminus\))  
        Set of valuations for which  \(\varphi\setminus_B\psi\) holds =  
        \(\{\nu \mid \varphi(\nu)=\top,\;\psi(\nu)=\bot\}\).  
        We can take \(\min\) of their security levels.
        Monus of security semiring?????
        \[
            a\setminus_S b =
            \begin{cases}
                a, & a<b,\\
                \ell_{\max}, &\text{otherwise}.
            \end{cases}
        \]
\end{itemize}

---

 5. Smalll Example

Let \(X=\{x,y\}\) with \(\lambda(x)=\mathrm{Pub}\), \(\lambda(y)=\mathrm{Sec}\),
and ordering \(\mathrm{Pub}<\mathrm{Sec}<\mathrm{Top}\).

- \(\varphi = x\):  
  Worlds: \((1,0)\to\mathrm{Pub}\), \((1,1)\to\mathrm{Sec}\).  
  \(\Phi(x)=\min(\mathrm{Pub},\mathrm{Sec})=\mathrm{Pub}.\)

- \(\psi = y\):  
  Worlds: \((0,1),(1,1)\to\mathrm{Sec}\).  
  \(\Phi(y)=\mathrm{Sec}.\)

- \(\varphi\vee\psi\): worlds \((1,0),(0,1),(1,1)\) → clearances Pub, Sec, Sec → min = Pub.

- \(\varphi\wedge\psi\): only \((1,1)\) → clearance max(Pub, Sec)=Sec.

- \(\varphi\setminus_B\psi\): world \((1,0)\) only → Pub, matching \(\mathrm{Pub}\setminus_S\mathrm{Sec}=\mathrm{Pub}.\)

