\section{Problem Definition}
Let's assume that any selection on the result of an aggregation is an 
atomic selection of the form .k=C or .k=.j. If not, we can write:

$\sigma_{\phi\land\psi}(q)=\sigma(\phi)(\sigma(\psi)(q))$
$\sigma_{\phi\lor\psi}(q)=\sigma(\phi)(q)\cup\sigma(\psi)(q)$
$\sigma_{\lnot\phi}(q)=q-\sigma(\phi)(q)$\\

$
\langle\sigma_{\#k\ op\ C}(q)\rangle^{\hat I}
=\{\{(u,\alpha\otimes[u.k\ op\ 1*C])\mid(u,\alpha)\in \langle q\rangle^{\hat I}, \alpha\otimes[u.k\ op\ C]\neq 0\}\}
$\\

$
\langle\sigma_{\#k\ op\ \#j}(q)\rangle^{\hat I}
=\{\{(u,\alpha\otimes[u.k\ op\ u.j])\mid(u,\alpha)\in \langle q\rangle^{\hat I}, \alpha\otimes[u.k\ op\ \#j]\neq 0\}\}
$\\
when .k refers to a column that dynamically has type “aggregate”.

% Semantics of [semimodule op semimodule] in different algebraic structures:
% - for formula, write it with concatenation
% - for Boolean function semiring, semantics of possible worlds
% - for why-semiring?
% - for which-semiring/lineage?
% - for counting-semiring?
% - for min-max/security semiring?
% - for tropical semiring?
% - etc.

What would be nice is that simple cases reduce to just evaluating joins.

$\langle\Pi_{\#1}(\sigma_{\#2>1}(\gamma_{\#1}[1:+](R)))\rangle^{\hat I}$
should be equal to something like
$\langle \epsilon(\Pi_{\#1}(\sigma_{\#2\neq\#4}(R\bowtie_{\#1=\#3} R))\rangle^{\hat I}$

R is assumed to be a binary relation without duplicates.


\paragraph{Reducing HAVING COUNT$(*) > C$ to a JOIN-equivalent query}

$R \subseteq  A \times B$ is a binary relation without duplicates.
Counting semiring : $K = (\mathcal{N}, +, ., 0, 1)$

Define $n_g = |\{b \in B\ \mid (g,b) \in R\}|$ to be the no. of unique tuples corresponding to group $g$.

$Q_1$:$\langle\Pi_{\#1}(\sigma_{\#2>C}(\gamma_{\#1}[1:+](R)))\rangle^{\hat I}$

$Q_2$: $\langle \epsilon(\Pi_{\#1}(\sigma_{distinct}(\rho_{R_1}(R)\bowtie_{\#1=\#3} \rho_{R_2}(R)\bowtie_{\#1=\#5}\dots\bowtie_{\#1=\#(2C+1)}\rho_{R_{C+1}}(R))))\rangle^{\hat I}$

Evaluating provenance semantics for $Q_1$:

$\gamma_{\#1}[1:+](R)$ evaluates $COUNT(*)$ for each corresponding group. $\sigma_{\#2>C}$ gives us groups with more than $C$ tuples, projecting on the group key $\Pi_{\#1}$.
Hence we get :

$\{ (g,\#1) \mid n_g > C \}$

For each group $g$ after selection ,
$$Prov(g) = \begin{cases}
   1, & n_g > C \\
   0, & \text{otherwise}
   \end{cases}$$

Evaluating provenance semantics for $Q_2$:

So we have a $C+1$-way self-join.
We keep the tuples for which  all of $(g,r_1),(g,r_2),\dots,(g,r_{C+1})$ are pairwise distinct. We then project on $\#1$ and elimate duplicates using $\epsilon$.

This basically gives us:

$\{(g,\#1) \mid \exists r_1,\dots,r_{C+1} \text{  s.t. } (g,r_i) \in R, r_i\neq r_j \forall i \neq j  \}$

We defined R to be a binary relation without duplicates, hence $$\exists r_1,\dots,r_{C+1} \text{  s.t. } (g,r_i) \in R, r_i\neq r_j \forall i \neq j $$ is the same as $$n_g>C$$


For $C+1$ self-joins we consider tuples with only distinct $r_i$s.
No. of such tuples $n_{r, g}= n_g \cdot (n_g-1) \cdots (n_g - C)$
And each joined tuples contribute 1. 
We remove duplicates with $\epsilon$

$$Prov(g) = \begin{cases}
   1, & \text{if } n_{r, g} > 0 \leftrightarrow n_g > C \\
   0, & \text{otherwise}
   \end{cases}$$