\documentclass{article}
\usepackage{amsmath,amssymb,amsthm}

\newtheorem{proposition}{Proposition}

\begin{document}

\section*{Detailed Proof of the Semiring Homomorphism}

Let
\[
  \mathbb{B} = (\{0,1\},\,\vee,\,\wedge,\,0,\,1)
\]
be the Boolean semiring, and let
\[
  \mathcal{B}(X) = \bigl(\{\,f:X\to\{0,1\}\},\,\vee_{\mathrm{pt}},\,\wedge_{\mathrm{pt}},\,0_X,\,1_X\bigr)
\]
be the semiring of Boolean functions on a set $X$, where for all $f,g\in\mathcal{B}(X)$ and all $x\in X$:
\[
  (f\vee_{\mathrm{pt}} g)(x) = f(x)\vee g(x),
  \quad
  (f\wedge_{\mathrm{pt}} g)(x) = f(x)\wedge g(x),
\]
and $0_X(x)=0$, $1_X(x)=1$ for every $x\in X$.

\begin{proposition}
Define
\[
  \phi : \mathbb{B} \;\longrightarrow\; \mathcal{B}(X)
  \quad\text{by}\quad
  \phi(0) = 0_X,
  \quad
  \phi(1) = 1_X.
\]
Then $\phi$ is a semiring homomorphism.
\end{proposition}

\begin{proof}
We check each requirement in turn.

---

\section*{Identity Preservation}

- Additive identity.  In $\mathbb B$, $0$ is the additive identity.  In $\mathcal B(X)$, $0_X$ is the additive identity.  By definition, $\phi(0)=0_X$.

- Multiplicative identity.  In $\mathbb B$, $1$ is the multiplicative identity.  In $\mathcal B(X)$, $1_X$ is the multiplicative identity.  By definition, $\phi(1)=1_X$.

Thus $\phi$ sends identities to identities.

---

\section{Preservation of Addition ($\vee$)}

Let $a,b\in\{0,1\}$.  We need to show
\[
  \phi(a\vee b)
  \;=\;
  \phi(a)\,\vee_{\mathrm{pt}}\,\phi(b).
\]
By definition of $\phi$, the left-hand side is the constant-$(a\vee b)$ function:
\[
  \phi(a\vee b) \;=\; 
  \bigl(x\mapsto a\vee b\bigr).
\]
On the right-hand side, recall $\phi(a)$ and $\phi(b)$ are the constant-$a$ and constant-$b$ functions, respectively:
\[
  \phi(a)(x) = a,
  \quad
  \phi(b)(x) = b,
  \quad
  \text{for all }x\in X.
\]
Then by the definition of pointwise $\vee_{\mathrm{pt}}$,
\[
  \bigl[\phi(a)\,\vee_{\mathrm{pt}}\,\phi(b)\bigr](x)
  = \phi(a)(x)\;\vee\;\phi(b)(x)
  = a \;\vee\; b.
\]
Hence the function $\phi(a)\vee_{\mathrm{pt}}\phi(b)$ is ALSO the constant-$(a\vee b)$ function.  Concretely:
\[
  \phi(a\vee b)(x)
  = (a\vee b)
  = \phi(a)(x)\vee \phi(b)(x)
  = \bigl[\phi(a)\vee_{\mathrm{pt}}\phi(b)\bigr](x),
  \quad \forall x\in X.
\]
Since two functions that agree on every $x\in X$ are equal, we conclude
\[
  \phi(a\vee b)
  = \phi(a)\,\vee_{\mathrm{pt}}\,\phi(b).
\]

\section{Preservation of Multiplication ($\wedge$)}

Let again $a,b\in\{0,1\}$.  We must show
\[
  \phi(a\wedge b)
  \;=\;
  \phi(a)\,\wedge_{\mathrm{pt}}\,\phi(b).
\]
- Left side: by definition, $\phi(a\wedge b)$ is the constant-$(a\wedge b)$ function:
  \[
    \phi(a\wedge b)(x) = a\wedge b,\quad\forall x\in X.
  \]
- Right side: since $\phi(a)$ is constant-$a$ and $\phi(b)$ is constant-$b$, the pointwise AND gives
  \[
    \bigl[\phi(a)\,\wedge_{\mathrm{pt}}\,\phi(b)\bigr](x)
    = \phi(a)(x)\;\wedge\;\phi(b)(x)
    = a\wedge b,
    \quad \forall x\in X.
  \]
Thus $\phi(a\wedge b)$ and $\phi(a)\wedge_{\mathrm{pt}}\phi(b)$ are the same constant-$(a\wedge b)$ function, and so
\[
  \phi(a\wedge b)
  = \phi(a)\,\wedge_{\mathrm{pt}}\,\phi(b).
\]

---

Since $\phi$ maps $0\mapsto0_X$, $1\mapsto1_X$, and preserves both addition and multiplication, it satisfies the definition of a semiring homomorphism.
\end{proof}

\end{document}
